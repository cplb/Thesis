%\chapter*{Summary}
\chapter{Summary}
\markboth{Summary}{}
\label{summary}

Gravitation is the dominant influence in most astrophysical interactions. Weak-field interactions have been extensively studied, but the strong-field regime remains unexplored. Gravitational waves (GWs) are an excellent means of accessing strong-field regions. We investigate what we can learn about both astrophysics and gravitation from strong-field tests and, in particular, GWs; we focus upon extreme-mass-ratio (EMR) systems where a small body orbits a much more massive one.

EMR bursts, a particular class of GW signals, could be used to determine the properties of massive black holes (MBHs). They could be detectable with a space-borne interferometer from many nearby galaxies, as well as the Galactic centre. Bursts could provide insightful constraints on the MBHs' parameters. These could elucidate the formation history of the MBHs and, by association, their host galaxies. The Galactic centre is the most promising source. Its event rate is determined by the stellar distribution surrounding the MBH; the rate is not high, but we still expect to gain useful astronomical information from bursts.

Strong-field tests may reveal deviations from general relativity (GR). We calculate modifications that could be observed assuming metric $f(R)$-gravity as an effective alternative theory. Gravitational radiation is modified, as are planetary precession rates. Both give a means of testing GR. However, existing laboratory measurements already place tight constraints on $f(R)$-gravity unless there exists a screening effect, such as the chameleon mechanism.

To make precision measurements of astrophysical systems or place exacting bounds on deviations from GR, we must have accurate GW templates. Transient resonances are currently not included in the prescription for generating EMR inspiral waveforms. Their effects can be estimated from asymptotic expansions of the evolving orbital parameters. The quantitative impact on parameter estimation has yet to be calculated, but it appears that it shall be necessary to incorporate resonances when creating inspiral waveforms.

\chapter{Summary}
\markboth{Summary}{}
\label{summary}

Gravitation is the dominant influence in most astrophysical interactions. To date, gravity has been exhaustively tested in weak-fields, but the strong-field regime remains unexplored. Gravitational waves (GWs) are an excellent means of accessing strong-field regions. We investigate what we can learn about both astrophysics and gravitation from strong-field tests and, in particular, GWs.

Extreme-mass-ratio bursts, a particular class of GW signals, could be used to determine the properties of the massive black holes. They could be detectable with a space-borne detector from many nearby galaxies, as well as the Galactic centre. Bursts could provide insightful constraints on the black holes' parameters. These could elucidate the formation history of the black holes and, by association, their host galaxies. The event rate is not high, but we still expect to gain useful astronomical information from bursts.

Strong-field tests may reveal deviations from general relativity (GR). We calculate modifications that could be observed assuming metric $f(R)$-gravity as an effective alternative. Gravitational radiation is modified, as our planetary precession rates. However, existing laboratory measurements already place tight constraints on $f(R)$-gravity, unless there exists a screening effect, such as the chameleon mechanism. We may be confident that GR provides an accurate description of GWs.

To make precision measurements of astrophysical systems or place exacting bounds on deviations from GR, we must have accurate GW templates. Transient resonances are currently not included in the prescription for generating extreme-mass-ratio inspiral waveforms. However, they could have a significant effect on the evolution of these systems. Their effects can be estimated from asymptotic expansions of the evolution of the system. The quantitative impact on parameter estimation has yet to be calculated, but it appears that it shall be necessary to incorporate a model for resonances when creating inspiral waveforms.
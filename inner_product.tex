\chapter{The signal inner product}\label{ap:inner-prod}

We wish to derive an inner product over the space of signals; we shall denote the product of signals $g$ and $h$ as $\innerprod{g}{h}$.

\section{Preliminaries}

\subsection{The Fourier transform}

We begin with some basic properties of Fourier transform \citep[section 13.1]{Riley2002}. We define transformations
\begin{subequations}
\begin{align}
x(t) = {} & \intd{-\infty}{\infty}{\widetilde{x}(f)\exp(2\pi ift)}{f} \\
\widetilde{x}(f) = {} & \intd{-\infty}{\infty}{x(t)\exp(-2\pi ift)}{t}.
\end{align}
\end{subequations}
The Dirac delta-function arises as
\begin{equation}
\delta(f) = \intd{-\infty}{\infty}{\exp(-2\pi ift)}{t}.
\end{equation}

We shall use Plancherel's theorem which proves the unitarity of the Fourier transformation
\begin{align}
\intd{-\infty}{\infty}{\left|x(t)\right|^2}{t} = {} & \int_{-\infty}^\infty\,\dd t \intd{-\infty}{\infty}{\widetilde{x}(f)\exp(2\pi ift)}{f} \intd{-\infty}{\infty}{\widetilde{x}^\ast(f')\exp(-2\pi if't)}{f'} \nonumber \\
% = {} & \int_{-\infty}^\infty\,\dd f \int_{-\infty}^\infty\,\dd f' \widetilde{x}(f) \widetilde{x}^\ast(f') \delta(f' - f) \nonumber \\
 = {} & \intd{-\infty}{\infty}{\left|\widetilde{x}(f)\right|^2}{f}.
\end{align}

\subsection{Wiener--Khinchin theorem}

We begin by deriving the Wiener--Khinchin theorem \citep[chapter 28]{Kittel1958}. For a real signal we have $\widetilde{x}(f) = \widetilde{x}^\ast(f)$, and since $\widetilde{x}(f) = \widetilde{x}^\ast(-f)$,
\begin{equation}
\left|\widetilde{x}(f)\right|^2 = \left|\widetilde{x}(-f)\right|^2.
\end{equation}
We use $\langle\ldots\rangle$ to denote time averaging, then
\begin{equation}
\left\langle x^2\right\rangle = \lim_{T \rightarrow \infty} \recip{2T} \intd{-T}{T}{\left[x(t)\right]^2}{t}.
\end{equation}
Applying Plancherel's theorem for our real signal
\begin{align}
\left\langle x^2\right\rangle = {} & \lim_{T \rightarrow \infty} \recip{2T} \intd{-\infty}{\infty}{\left|\widetilde{x}(f)\right|^2}{f} = \lim_{T \rightarrow \infty} \recip{T} \intd{0}{\infty}{\left|\widetilde{x}(f)\right|^2}{f}.
\end{align}

The power spectrum $G(f)$ is
\begin{equation}
G(f) = \lim_{T \rightarrow \infty} \recip{T} \overline{\left|\widetilde{x}(f)\right|^2},
\end{equation}
where an overline represents an ensemble average. Therefore
\begin{equation}
\overline{\left\langle x^2\right\rangle} = \intd{0}{\infty}{G(f)}{f}.
\end{equation}
If $x(t)$ is a randomly varying signal we can use the ergodic principle to equate a time average with an ensemble over multiple realisations. Hence $\overline{\left\langle x^2\right\rangle} = \left\langle x^2\right\rangle$ and we can drop the overline.

The correlation function for a random process is
\begin{align}
C(\tau) = {} & \left\langle x(t)x(t + \tau)\right\rangle \\
% = {} & \lim_{T \rightarrow \infty} \recip{2T} \intd{-T}{T}{x(t)x(t + \tau)}{t} \nonumber \\
 = {} & \lim_{T \rightarrow \infty} \recip{2T} \int_{-T}^{T}\,\dd t \intd{-\infty}{\infty}{\widetilde{x}(f)\exp(2\pi ift)}{f} \intd{-\infty}{\infty}{\widetilde{x}(f')\exp\left[2\pi if'(t + \tau)\right]}{f'} \nonumber \\
% = {} & \lim_{T \rightarrow \infty} \recip{2T} \int_{-\infty}^{\infty}\,\dd f \int_{-\infty}^{\infty}\,\dd f' \, \widetilde{x}(f)\widetilde{x}(f')\exp(2\pi if'\tau)\delta(f + f') \nonumber \\
 = {} & \lim_{T \rightarrow \infty} \recip{2T} \intd{-\infty}{\infty}{\left|\widetilde{x}(f)\right|^2\exp(2\pi if\tau)}{f}.
\end{align}
We can rewrite this in terms of the power spectrum
\begin{align}
C(\tau) = {} & \recip{2} \intd{-\infty}{\infty}{G(f)\exp(2\pi if\tau)}{f} = \intd{0}{\infty}{G(f)\cos(2\pi f\tau)}{f}.
\end{align}
Inverting these
\begin{align}
G(f) = {} & 2 \intd{-\infty}{\infty}{C(\tau)\exp(-2\pi i f\tau)}{\tau} = 4 \intd{0}{\infty}{C(\tau) \cos(2\pi f\tau)}{\tau}.
\end{align}
The power spectrum and correlation function are related to each other by the Fourier transform. This is the Wiener--Khinchin theorem.

\section{Defining the inner product}

\subsection{Gaussian noise}

We consider a normally distributed noise signal $n(t)$ with zero mean and standard deviation $\sigma_n$. The variance is
\begin{equation}
\left\langle n^2 \right\rangle = C_n(0) = \sigma_n^2,
\end{equation}
introducing correlation function $C_n(\tau)$. If we have a measured signal $s(t)$ and a true signal $h(t)$, the probability $p(s|h)$ is that of the realisation of noise such
\begin{equation}
s = h + n.
\end{equation}

Let us consider a discrete signal $n_i \equiv n(t_i)$, with $t_i - t_j = (i - j)\upDelta t$ $\{i,j = -N, \ldots, N\}$ and $\upDelta T = 2T/(2N + 1)$. For a single point \citep{Finn1992}:
\begin{equation}
p(s_i|h_i) = \recip{\sqrt{2\pi C_n(0)}} \exp\left[-\recip{2} \frac{n_i^2}{C_n(0)}\right].
\end{equation}
Expanding this to the entire signal
\begin{equation}
p(s|h) = \recip{\sqrt{(2\pi)^{2N+1} \det C_{n,\,ij}}} \exp\left[-\recip{2} \sum_{k,l}C^{-1}_{kl}n_k n_l\right],
\label{eq:prod_prob}
\end{equation}
introducing short-hand $C_{n,\,ij} \equiv C_n(t_i - t_j)$ and defining the inverse matrix $C^{-1}_{kl}$ such that
\begin{equation}
\delta_{jl} = \sum_l C_{n,\,jk}C^{-1}_{kl}.
\end{equation}

To transform to the continuum (and infinite duration) limit we identify
\begin{equation}
\lim_{T \rightarrow \infty; \; \upDelta t \rightarrow 0} \sum_j \upDelta t \rightarrow \lim_{T \rightarrow \infty} \int_{-T}^{T}\,\dd t_j.
\end{equation}
To change between Kronecker and Dirac deltas
\begin{equation}
\sum_j \delta_{jk} = \intd{-T}{T}{\delta(t_j - t_k)}{t_j},
\end{equation}
hence
\begin{equation}
\delta(t_j - t_k) = \lim_{\upDelta t \rightarrow 0} \recip{\upDelta t} \delta_{jk}.
\end{equation}

Using the inverse matrix definition
\begin{align}
\exp(-2\pi i f t_k) = {} & \sum_j \exp(-2\pi ift_j) \delta_{jk} \nonumber \\
% = {} & \sum_j \exp(-2\pi ift_j) \sum_l C_{n,\,jl}C^{-1}_{lk} \nonumber \\
 = {} & \recip{(\upDelta t)^2} \sum_j \upDelta t \exp(-2\pi ift_j) \sum_l \upDelta t \, C_{n,\,jl}C^{-1}_{lk}.
\end{align}
Taking the limit
\begin{align}
\exp(-2\pi i f t_k) = {} & \lim_{T \rightarrow \infty; \; \upDelta t \rightarrow 0} \recip{(\upDelta t)^2} \intd{-T}{T}{\exp(-2\pi ift_j)}{t_j} \intd{-T}{T}{C_n(t_j-t_l)C^{-1}(t_l,t_k)}{t_l} \nonumber \\
 = {} & \lim_{\upDelta t \rightarrow 0} \recip{(\upDelta t)^2} \intd{-\infty}{\infty}{C_n(\tau)\exp(-2\pi if\tau)}{\tau} \intd{-\infty}{\infty}{C^{-1}(t_l,t_k)\exp(-2\pi if t_l)}{t_l},
\end{align}
where $\tau = t_j - t_l$. Defining the transformation
\begin{equation}
\widetilde{C^{-1}}(f,t_k) = \intd{-\infty}{\infty}{C^{-1}(t,t_k)\exp(-2\pi if t)}{t},
\end{equation}
and using the Wiener-Khinchin theorem to the define power spectral density \citep{Cutler1994}
\begin{align}
\label{eq:PSD-WK}
S_n(f) = {} & \lim_{T \rightarrow \infty} \recip{T} \overline{\left|\widetilde{n}(f)\right|^2} \\
 = {} & 2 \intd{-\infty}{\infty}{C_n(\tau)\exp(-2\pi i f\tau)}{\tau},
\end{align}
we have
\begin{equation}
\exp(-2\pi i f t_k) = \lim_{\upDelta t \rightarrow 0} \recip{(\upDelta t)^2} \frac{S_n(f)}{2} \widetilde{C^{-1}}(f,t_k).
\end{equation}
This can be rearranged to define $\widetilde{C^{-1}}(f,t_k)$ \citep{Finn1992}.

The term in the exponential in \eqnref{prod_prob} has the limit
\begin{align}
\mathcal{H} = {} & \recip{2}\lim_{T \rightarrow \infty; \; \upDelta t \rightarrow 0} \sum_{j,k}C^{-1}_{jk}n_j n_k \nonumber \\
 = {} & \recip{2}\lim_{T \rightarrow \infty; \; \upDelta t \rightarrow 0} \recip{(\upDelta t)^2}\int_{-T}^{T}\,\dd t_j \int_{-T}^{T}\,\dd t_k \, C^{-1}(t_j,t_k) n(t_j)n(t_k) \nonumber \\
% = {} & \recip{2}\lim_{\upDelta t \rightarrow 0} \recip{(\upDelta t)^2}\int_{-\infty}^{\infty}\,\dd t_j \int_{-\infty}^{\infty}\,\dd t_k \left\{ \intd{-\infty}{\infty}{\widetilde{C^{-1}}(f,t_k)\exp(2\pi if t_j)}{f} \right. \nonumber \\*
% {} & \times \left. \intd{-\infty}{\infty}{\widetilde{n}(f')\exp(2\pi i f't_j)}{f'}n(t_k)\right\} \nonumber \\
 = {} & \recip{2}\lim_{\upDelta t \rightarrow 0} \recip{(\upDelta t)^2} \int_{-\infty}^{\infty}\,\dd t_k \int_{-\infty}^{\infty}\,\dd f \, \widetilde{C^{-1}}(f,t_k)\widetilde{n}(-f)n(t_k) \nonumber \\
% = {} & \int_{-\infty}^{\infty}\,\dd t_k \int_{-\infty}^{\infty}\,\dd f \, \frac{\exp(-2\pi i f t_k)}{S_n(f)}\widetilde{n}^\ast(f)n(t_k) \nonumber \\
 = {} & \intd{-\infty}{\infty}{\frac{\widetilde{n}^\ast(f)\widetilde{n}(f)}{S_n(f)}}{f} \nonumber \\
 = {} & \recip{2}\innerprod{n}{n},
 \end{align}
defining the inner product
\begin{align}
\innerprod{g}{h} = {} & 2\intd{-\infty}{\infty}{\frac{\widetilde{g}^\ast(f)\widetilde{h}(f)}{S_n(f)}}{f} = 2 \intd{0}{\infty}{\frac{\widetilde{g}^\ast(f)\widetilde{h}(f) + \widetilde{g}(f)\widetilde{h}^\ast(f)}{S_n(f)}}{f}.
\end{align}
This is a noise-weighted inner product over the space of real signals. The probability of the signal is
\begin{equation}
p(s|h) \propto \exp\left[-\recip{2}\innerprod{n}{n}\right].
\end{equation}

\subsection{Noise power spectral density}

We have defined the power spectral density $S_n(f)$ using the Wiener-Khinchin theorem. It is more usual to define it in terms of the average of the noise spectrum. We shall again appeal to the ergodic principle to equate noise time and ensemble averages. Averaging two frequency components
\begin{align}
\left\langle\widetilde{n}(f)\widetilde{n}^\ast(f')\right\rangle = {} & \lim_{T \rightarrow \infty} \recip{2T} \intd{-T}{T}{\widetilde{n}(f)\exp(2\pi i f \tau)\widetilde{n}^\ast(f')\exp(2\pi i f' \tau)}{\tau} \nonumber \\
 = {} & \lim_{T \rightarrow \infty} \recip{2T} \widetilde{n}(f)\widetilde{n}^\ast(f')\delta(f-f').
\end{align}
Taking the ensemble average of both sides, and exploiting the properties of the delta-function, we can use \eqnref{PSD-WK} to identify
\begin{equation}
\left\langle\widetilde{n}(f)\widetilde{n}^\ast(f')\right\rangle = \recip{2} S_n(f)\delta(f-f').
\end{equation}
This can serve as a definition for the noise power spectral density.

%\subsection{Properties of the inner product}

%Consider an ensemble average over multiple noise realisations, which is the same as a time average assuming stationarity of the noise spectrum:
%\begin{align}
%\left\langle\innerprod{n}{p}\innerprod{n}{q}\right\rangle = {} & \lim_{T \rightarrow \infty} \recip{2T} \intd{-T}{T}{\innerprod{n(t+\tau)}{p(t)}\innerprod{n(t+\tau)}{q(t)}}{\tau} \nonumber \\
%% = {} & \lim_{T \rightarrow \infty} \frac{2}{T} \int_{-T}^{T}\,\dd\tau \left\{ \intd{-\infty}{\infty}{\frac{\widetilde{n}^\ast(f)\exp(-2\pi if\tau)\widetilde{p}(f)}{S_n(f)}}{f} \right. \nonumber \\
%%  {} & \times \left. \intd{-\infty}{\infty}{\frac{\widetilde{n}^\ast(f')\exp(-2\pi if'\tau)\widetilde{q}(f')}{S_n(f')}}{f'}\right\} \nonumber \\
% = {} & \lim_{T \rightarrow \infty} \frac{2}{T} \intd{-\infty}{\infty}{\frac{\widetilde{n}^\ast(f)\widetilde{n}(f)}{S_n(f)}\frac{\widetilde{p}(f)\widetilde{q}^\ast(f)}{S_n(f)}}{f} \nonumber \\
% = {} & \innerprod{p}{q},
%\end{align}
%using the definition of the noise spectrum to obtain the final line.


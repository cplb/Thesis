\chapter{Energy change in two-body encounters}\label{ap:Chandra}

Two-body encounters are common in dense stellar systems, such as those found in galactic centres. During these encounters, energy and angular momentum can be exchanged. It is this that leads to the relaxation of a stellar system. \citet[chapter 2]{Chandrasekhar1960} derived the squared change in energy expected in a two-body encounter. We reproduce this here for use in \chapref{relax}, where we discuss the relaxation time-scale for the Galactic NSC.

We consider the interaction of a field star, denoted by 1, with a test star, 2; the centre-of-gravity and relative velocities are
\begin{subequations}
\label{eq:Vs}
\begin{align}
\boldsymbol{V}\sub{g} = {} & \recip{m_1 + m_2}\left(m_1 \boldsymbol{v}_1 + m_2 \boldsymbol{v}_2\right),\\
\boldsymbol{V} = {} & \boldsymbol{v}_1 - \boldsymbol{v}_2.
\end{align}
\end{subequations}
Hence
\begin{subequations}
\begin{align}
v_1^2 = {} & V\sub{g}^2 - 2\dfrac{m_2}{m_1 + m_2}V\sub{g}V \cos\Phi + \left(\dfrac{m_2}{m_1 + m_2}\right)^2V^2,\\
v_2^2 = {} & V\sub{g}^2 + 2\dfrac{m_1}{m_1 + m_2}V\sub{g}V \cos\Phi + \left(\dfrac{m_1}{m_1 + m_2}\right)^2V^2,
\end{align}
\end{subequations}
where $\Phi$ is the angle between $\boldsymbol{V}\sub{g}$ and $\boldsymbol{V}$, and
\begin{subequations}
\label{eq:V2s}
\begin{align}
V\sub{g}^2 = {} & \recip{(m_1 + m_2)^2}\left(m_1v_1^2 + m_2v_2^2 + 2 m_1 m_2 v_1 v_2 \cos\theta\right),\\
V^2 = {} & v_1^2 + v_2 - 2 v_1 v_2 \cos\theta,
\end{align}
\end{subequations}
where $\theta$ is the angle between $\boldsymbol{v}_1$ and $\boldsymbol{v}_2$. The change in energy of the test star during the interaction is
\begin{align}
\Delta E = {} & \recip{2} m_2 \left({v'_2}^2 - v_2^2\right)\\
 = {} & \dfrac{m_1 m_2}{m_1 + m_2}V\sub{g}V\left(\cos\Phi' - \cos\Phi\right),
\end{align}
using primed variables for values after the interaction, and unprimed ones for values before. If we project the angle onto the orbital plane
\begin{equation}
\Delta E = \dfrac{m_1 m_2}{m_1 + m_2}V\sub{g}V\left(\cos\phi' - \cos\phi\right)\cos i,
\end{equation}
where $\phi$ is the angle in the plane, and $i$ is the inclination of $\boldsymbol{V}\sub{g}$ out of the plane. We define the deflection angle $\psi$ such that
\begin{equation}
\phi' - \phi = \pi - 2\psi,
\end{equation}
hence
\begin{equation}
\Delta E = -2\dfrac{m_1 m_2}{m_1 + m_2}V\sub{g}V\cos(\phi - \psi)\cos\psi\cos i.
\end{equation}

We now need to calculate the encounter rate. This requires us to know the number of field stars per unit volume and per velocity space element. We make the simplifying assumption that the density of stars is uniform. This is not the case for the NSC; however, we approximate it as so in order to make the problem tractable. Using an averaged value, the number of stars is
\begin{equation}
\dd N = n(v_1, \theta, \varphi)\,\dd v_1 \,\dd \theta \,\dd \varphi \,\dd^3r,
\end{equation}
using spherical polar coordinates for velocity space. Introducing $D$ as the impact parameter for the encounter and $\Theta$ for the angle between the fundamental plane (containing $\boldsymbol{v}_1$ and $\boldsymbol{v}_2$) and the orbital plane, the number of events in time interval $\delta t$ is
\begin{equation}
\dd \Gamma =  n(v_1, \theta, \varphi)\,\dd v_1 \,\dd \theta \,\dd \varphi \dfrac{\dd \Theta}{2\pi} 2\pi D \,\dd D \,V \delta t.
\end{equation}
The squared change in energy for these encounters is
\begin{align}
\Delta E^2(v_1,\theta,\varphi,\Theta,D) = {} & \left(\Delta E\right)^2\,\dd \Gamma\\
 = {} & 4 n(v_1,\theta,\varphi)V\sub{g}^2V^3\left(\dfrac{m_1m_2}{m_1+m_2}\right)^2\cos^2i\cos^2(\phi-\psi)\cos^2\psi D \nonumber \\*
 {} & \times \left. \dd v_1\,\dd\theta\,\dd\varphi\,\dd\Theta\,\dd D\,\delta t. \right.
\end{align}
We must integrate out all these dependencies.

Since we have assumed the stellar density does not depend upon position, we can simply integrate over the impact parameter; this is related to the deflection angle by
\begin{equation}
\recip{\cos^2\psi} = 1 + \dfrac{D^2V^4}{G^2(m_1 + m_2)^2}.
\end{equation}
Thus
\begin{align}
\Delta E^2(v_1,\theta,\varphi,\Theta,\psi) =  {} & 4 n(v_1,\theta,\varphi)\dfrac{V\sub{g}^2}{V}G^2m_1^2 m_2^2\cos^2i\dfrac{\cos^2(\phi-\psi)\sin\psi}{\cos\psi} \nonumber \\*
 {} & \times \left. \dd v_1\,\dd\theta\,\dd\varphi\,\dd\Theta\,\dd \psi\,\delta t. \right.
\end{align}
The integral over $\psi$ is
\begin{align}
I(\psi_0) = {} & \intd{0}{\psi_0}{\dfrac{\cos^2(\phi-\psi)\sin\psi}{\cos\psi}}{\psi}\\
 = {} & \dfrac{\sin 2\phi}{2}\left(\psi_0 - \dfrac{\sin 2\psi_0}{2}\right) - \dfrac{\cos 2\phi}{2}\left(\dfrac{\cos 2\psi_0}{2}\right) - \sin^2\phi\ln(\cos\psi_0).
\end{align}
Naively we might think the upper limit for the deflection limit should be $\psi_0 = \pi/2$; however, this would introduce a logarithmic divergence. In actuality there is a physical cut-off, reflecting a finite bound for the maximum impact parameter $D_0$ \citep{Weinberg1986}. This is set by the scale of the system, beyond which scattering is negligible. While the logarithmic term is finite, it is still large, dominating the other terms which are $\order{1}$; we therefore neglect the subdominant terms,
\begin{equation}
\Delta E^2(v_1,\theta,\varphi,\Theta) \simeq  4 n(v_1,\theta,\varphi)\dfrac{V\sub{g}^2}{V}G^2m_1^2 m_2^2\cos^2i\sin^2\phi\ln\left(\recip{\cos\psi_0}\right)\,\dd v_1\,\dd\theta\,\dd\varphi\,\dd\Theta\,\delta t.
\end{equation}

Next, we integrate over the orbital plane inclination using
\begin{equation}
\cos i\sin\phi = \sin\Phi\cos\Theta,
\end{equation}
so that
\begin{equation}
\Delta E^2(v_1,\theta,\varphi) \simeq  4\pi n(v_1,\theta,\varphi)\dfrac{V\sub{g}^2}{V}G^2m_1^2 m_2^2\sin^2\Phi\ln\left(\recip{\cos\psi_0}\right)\,\dd v_1\,\dd\theta\,\dd\varphi\,\delta t.
\end{equation}
We are now left with just the velocity variables.

An expression for $\sin^2\Phi$ can be obtained from \eqnref{Vs} and \eqnref{V2s}, after some rearrangement
\begin{equation}
\dfrac{V_g^2}{V}\sin^2\Phi = \dfrac{v_1^2v_2^2\sin^2\theta}{\left(v_1^2 + v_2^2 - 2v_1 v_2 \cos\theta\right)^{3/2}}.
\end{equation}
To proceed further we must specify the form of $n(v_1,\theta,\varphi)$, if we assume isotropy
\begin{equation}
n(v_1,\theta,\varphi) = n(v_1)\recip{4\pi}\sin\theta.
\end{equation}
The integral over $\varphi$ is then trivial,
\begin{align}
\Delta E^2(v_1,\theta) \simeq {} & \pi n(v_1)\dfrac{G^2m_1^2 m_2^2v_1^2v_2^2\sin^3\theta}{\left(v_1^2 + v_2^2 - 2v_1 v_2 \cos\theta\right)^{3/2}} \nonumber \\*
 {} & \times \left. \ln\left[1 + \dfrac{D_0\left(v_1^2 + v_2^2 - 2v_1 v_2 \cos\theta\right)^2}{G^2\left(m_1 + m_2\right)^2}\right]\,\dd v_1\,\dd\theta\,\delta t.\right.
\end{align}

To integrate over $\theta$ it is easier to recast in terms of $V$; the integral is
\begin{align}
J = {} & v_1^2 v_2^2 \intd{0}{\pi}{\dfrac{\sin^3\theta}{\left(v_1^2 + v_2^2 - 2v_1 v_2 \cos\theta\right)^{3/2}}\ln\left[1 + \dfrac{D_0\left(v_1^2 + v_2^2 - 2v_1 v_2 \cos\theta\right)^2}{G^2\left(m_1 + m_2\right)^2}\right]}{\theta} \\
 = {} & v_1 v_2 \intd{V_-}{V_+}{\dfrac{\sin^2\theta}{V^2}\ln\left(1 + q^2V^4\right)}{V},
\end{align}
where the limits are
\begin{equation}
V_+ = v_1 + v_2, \quad V_- = |v_1 - v_2|,
\end{equation}
and we have introduced
\begin{equation}
q = \dfrac{D_0}{G\left(m_1+m_2\right)}.
\end{equation}
Using \eqnref{V2s} to rearrange, and then integrating by parts gives
\begin{align}
J = {} & \recip{4 v_1 v_2} \intd{V_-}{V_+}{\dfrac{\left(V_+^2 - V^2\right)\left(V^2 - V_-^2\right)}{V^2}\ln\left(1 + q^2V^4\right)}{V} \\
 = {} & \recip{4 v_1 v_2} \left\{\left[\dfrac{3V_+^2V_-^2 + 3\left(V_+^2 + V_-^2\right)V^2 - V^4}{3V} \ln\left(1 + q^2V^4\right)\right]^{V_+}_{V_-} \right. \nonumber \\* 
 {} & - \left. \vphantom{\left[\dfrac{\left(V_+^2\right)V^2}{3V}\right]^{V_+}_{V_-}} \intd{V_-}{V_+}{\dfrac{3V_+^2V_-^2 + 3\left(V_+^2 + V_-^2\right)V^2 - V^4}{3V}\dfrac{4q^2V^3}{1+ q^2V^4}}{V}\right\};
\end{align}
the former piece still contains the logarithmic term which we know must be large. It is therefore the dominant piece of the integral and we neglect the latter \citep{Chandrasekhar1941a},
\begin{equation}
J \simeq \recip{6 v_1 v_2} \left[\left(3V_-^2 + V_+^2\right)V_+\ln\left(1 + q^2V_+^4\right) - \left(3V_+^2 + V_-^2\right)V_-\ln\left(1 + q^2V_-^4\right)\right].
\end{equation}
This may be further simplified, reusing the limit of large $q$ previously employed to neglect terms which do not contain its logarithm \citep{Chandrasekhar1941a,Chandrasekhar1941b},
\begin{align}
J \simeq {} & \recip{3 v_1 v_2} \left[\left(3V_-^2 + V_+^2\right)V_+\ln\left(qV_+^2\right) - \left(3V_+^2 + V_-^2\right)V_-\ln\left(qV_-^2\right)\right] \\
 \simeq {} & \dfrac{4}{3v_1v_2}\begin{cases}
\left(v_1^3 + v_2^3\right)\ln\left[q\left(v_1 + v_2\right)^2\right] - \left(v_2^3 - v_1^3\right)\ln\left[q\left(v_1 - v_2\right)^2\right] & v_1 \leq v_2 \\
\left(v_1^3 + v_2^3\right)\ln\left[q\left(v_1 + v_2\right)^2\right] - \left(v_1^3 - v_2^3\right)\ln\left[q\left(v_1 - v_2\right)^2\right] & v_1 \geq v_2
\end{cases} \\
 \simeq {} & \dfrac{8}{3}\begin{cases}
\dfrac{v_2^2}{v_1}\ln\left(\dfrac{v_1 + v_2}{v_2 - v_1}\right) + \dfrac{v_1^2}{v_2}\left[\ln\left(qv_2^2\right) + \ln\left(1 - \dfrac{v_1^2}{v_2^2}\right)\right] & v_1 < v_2 \\
v_2\left[\ln\left(qv_2^2\right) + \ln 4\right] & v_1 = v_2\\
\dfrac{v_1^2}{v_2}\ln\left(\dfrac{v_1 + v_2}{v_1 - v_2}\right) + \dfrac{v_2^2}{v_1}\left[\ln\left(qv_2^2\right) + \ln\left(\dfrac{v_1^2}{v_2^2} - 1\right)\right] & v_1 > v_2
\end{cases} \\
\approx {} & \dfrac{8}{3}\begin{cases}
\dfrac{v_1^2}{v_2}\ln\left(qv_2^2\right)\vspace{1.0mm} & v_1 \leq v_2 \\
\dfrac{v_2^2}{v_1}\ln\left(qv_2^2\right) & v_1 \geq v_2
\end{cases}.
\end{align}
This form maintains the correct limit for $v_2 \rightarrow 0$.\footnote{As stressed by \citet{Antonini2011}, it is important to include both the piece for $v_1 \leq v_2$ and $v_1 \geq v_2$ in the final expressions.} We are left with
\begin{equation}
\Delta E^2(v_1) \simeq \dfrac{8\pi}{3} n(v_1)G^2m_1^2 m_2^2\ln\left(qv_2^2\right)\left\{\begin{array}{lr}\dfrac{v_1^2}{v_2}\vspace{1.0mm} & v_1 \leq v_2\\ \dfrac{v_2^2}{v_1} & v_1 \geq v_2 \end{array}\right\}\,\dd v_1\delta t.
\end{equation}
This is the change in energy squared for the test star as a function of its initial velocity.

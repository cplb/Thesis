\chapter{Gravitation}

\section{$f(R)$}

General relativity (GR) is a well tested theory of gravity~\cite{Will2006}; so far no evidence has been found that suggests it is not the correct classical theory of gravitation. However, there are many unanswered questions that remain regarding gravity which motivate the exploration of alternate theories: What are the true natures of dark matter and dark energy? How should we formulate a quantizable theory of gravity? What drove inflation in the early Universe?  Is GR the only theory that is consistent with current observations? Moreover, the majority of the tests that have been carried out to date have been in the weak-field, low-energy regime~\cite{Will2006, Will1993}: in the laboratory~\cite{Adelberger2009, Adelberger2003}, within the Solar System~\cite{Bertotti2003, Everitt2009} or using binary pulsars~\cite{Stairs2003}. It is not unreasonable to suppose that GR would begin to break down at higher energies.

Over the coming decade, a new avenue for testing relativity will be opened up, through the detection of gravitational waves (GWs) using the existing ground-based GW detectors, the Laser Interferometer Gravitational-Wave Observatory (LIGO)~\cite{Abramovici1992, Abbott2009}, Virgo~\cite{Accadia2010} and GEO~\cite{Willke2002, Abadie2010}, and the proposed space-based GW detector, the Laser Interferometer Space Antenna (LISA)~\cite{Bender1998, Danzmann2003}. These detectors will observe GWs generated during the inspiral and merger of binary systems comprising one or more black holes (BHs). The GWs are generated in the strong-field regime, while the components are highly relativistic and the spacetime is evolving dynamically: GW astronomy will open a new window into the strong-field regime of gravity, complementing traditional electromagnetic observations~\cite{Psaltis2008a}.  A comparison of the GWs observed from such systems with the predictions of GR will provide powerful tests of the theory in a region yet to be explored.

%The radiation generated during the final merger and ring-down of two BHs will offer tests of GR in the highest energy and most dynamical sector, but it is thought that the most sensitive tests will come from LISA observations of extreme-mass-ratio inspirals (EMRIs)~\cite{Amaro-Seoane2007}. An EMRI involves the inspiral of a stellar-mass compact object, a white dwarf, neutron star or BH, into a massive BH in the centre of a galaxy. The mass of the compact object is typically $1$--$10M_{\odot}$, while the mass of the massive BH (for sources in the LISA band) will be $\sim10^5$--$10^7 M_\odot$, so the mass-ratio is of the order of $\sim10^{-7}$--$10^{-4}$. This extreme mass-ratio means that the inspiral proceeds slowly, and on short time-scales the compact object acts like a test particle moving in the background spacetime of the central BH. LISA will detect $\sim10^5$ cycles of gravitational radiation generated while the compact object is in the strong field of the spacetime, and this encodes a detailed map of the spacetime structure outside the central BH. This idea was first elucidated by Ryan~\cite{Ryan1995c, Ryan1997a}, who showed that, for an arbitrary stationary and axisymmetric spacetime in GR, the multipole moments of the spacetime enter at different orders in an expansion of the frequency of small vertical or radial oscillations of circular, equatorial orbits. As these frequencies are in principle observable in the GWs generated during an inspiral, it should be possible to measure the multipole moments from an EMRI observation and hence test whether the central object is a Kerr BH: according to the no-hair theorem, a Kerr BH is described completely by its mass $M$ and spin angular momentum $J$~\cite{Israel1967, Israel1968, Carter1971, Hawking1972, Robinson1975}, and its mass multipole $M_l$ ($M_0 \equiv M$) and mass-current multipole moments $S_l$ ($S_1 \equiv J$) are determined from these according to~\cite{Hansen1974}
%\begin{equation}
%M_l + iS_l = M \left(i\frac{J}{M}\right)^l .
%\end{equation}
%The multipole expansion is not a convenient way to characterise arbitrary spacetimes, since the Kerr metric itself requires an infinite number of multipoles to fully characterise. Subsequent authors have instead adopted the approach of considering bumpy BH spacetimes~\cite{Collins2004, Glampedakis2006a, Barack2007, Gair2008a}, which deviate from the Kerr metric by a small amount and depend on some parameter, $\epsilon$, such that $\epsilon = 0$ is precisely the Kerr solution. Relatively small perturbations to the Kerr solution can be detected in EMRI observations due to small differences in the precession frequencies that accumulate over the $100\,000$ waveform cycles that will be detected. There are also certain qualitative features that could be smoking-guns for a departure from the Kerr metric, such as ergodicity in the orbits~\cite{Gair2008a}, persistent resonances~\cite{Lukes-Gerakopoulos2010} or a shift in the frequency of plunge~\cite{Kesden2005, Gair2008a}.

%The majority of the work to date has focused on spacetimes that are solutions in GR, but which deviate from the Kerr solution. However, if GR was not the correct theory of gravity, this could also lead to detectable signatures in the observed gravitational waves. Certain alternative theories of gravity, including $f(R)$, do admit the Kerr metric as a solution, since it has vanishing Ricci tensor, $R_{\mu\nu} = 0$~\cite{Psaltis2008, Yunes2011}. However, the Kerr metric need not be the expected end state of gravitational collapse~\cite{Barausse2008}. If a Kerr BH existed in an alternative theory, the geodesics would be the same, but the energy flux carried by the GWs could still be different, and so differences would show up in the rate of inspiral; although in many cases these differences do not appear at leading order. In most cases, however, either the Kerr metric is not admitted as a solution, or it is not the correct metric to describe collapsed objects~\cite{Yunes2011}. Waveform differences then show up as a result of the differences in the instantaneously-geodesic orbits of the compact object involved in the EMRI. Since the leading-order energy-momentum tensor of the GWs often takes the same form as in GR~\cite{Stein2011}, this is the primary effect and means the problem of testing alternative theories through EMRI observations is equivalent to the spacetime mapping programme within GR described previously.

%As a consequence of the difficulties of solving for GW emission in alternative theories, work on testing alternative theories of gravity using LISA EMRIs has so far been restricted to a few cases. In Brans-Dicke gravity, in which the gravitational field is coupled to a scalar field, differences show up due to a modification to the inspiral rate that arises from dipole radiation of the scalar field~\cite{Berti2005}. Neutron star EMRIs are required since the dipole radiation depends on a sensitivity difference between the two objects, and the sensitivity is the same for all BHs. Lower mass central BHs provide the most powerful constraints, but a LISA observation of a neutron star EMRI into a $10^4 M_{\odot}$ BH could place constraints on the Brans-Dicke coupling parameter that are competitive with Solar System constraints~\cite{Berti2005}. In dynamical Chern-Simons modified gravity, the action is modified by a parity-violating correction, inspired by string theory~\cite{Alexander2008, Alexander2009a}. In this case, the BH solution differs from the Kerr solution at the fourth multipole, $l = 4$, but the energy-momentum tensor of gravitational radiation takes the same form as in GR~\cite{Sopuerta2009a}. LISA observations of EMRIs should place constraints on the Chern-Simons coupling parameter that are an order of magnitude better than will be possible from binary pulsar observations, although a full analysis accounting for parameter degeneracies has not yet been carried out~\cite{Sopuerta2009a}. 

In this work, we focus our attention on metric $f(R)$-gravity, in which the Einstein-Hilbert action is modified by replacing the Ricci scalar $R$ with an arbitrary function $f(R)$. This is one of the simplest extensions to standard GR~\cite{Sotiriou2010, DeFelice2010}. It has attracted significant interest because the flexibility in defining the function $f(R)$ allows a wide range of cosmological phenomena to be described~\cite{Nojiri2007, Capozziello2007a}. For example, Starobinsky~\cite{Starobinsky1980} suggested that a quadratic addition to the field equations could drive exponential expansion of the early Universe~\cite{Vilenkin1985}: inflation in modern terminology. In this model $f(R) = R - R^2/(6\Upsilon^2)$; the size of the quadratic correction can be tightly constrained by considering the spectrum of curvature perturbations generated during inflation~\cite{Starobinskii1983, Starobinskii1985}. Using the results of the Wilkinson Microwave Anisotropy Probe~\cite{Jarosik2011, Larson2011}, the inverse length-scale can be constrained to $\Upsilon \simeq 3 \times 10^{-6} (50/N) l\sub{P}^{-1}$~\cite{Starobinsky2007, DeFelice2010}, where $N$ is the number of e-folds during inflation and $l\sub{P}$ is the Planck length. 

We consider simple $f(R)$ corrections within the framework of linearised gravity, and explore what constraints LISA might be able to place on the form of $f(R)$ (we will not consider cosmological implications where terms beyond linear order could play a significant role). We will see that, although the field equations for $f(R)$-gravity do admit the Kerr metric as a solution~\cite{Psaltis2008, Barausse2008}, this is not necessarily the metric that describes the exterior of collapsed objects. We consider the modifications to geodesic orbits in the weak-field of the $f(R)$ spacetime exterior to massive objects and, assuming this also describes the weak-field external to a BH, we estimate how observable the differences in the precession frequencies will be by LISA. We will also describe Solar System and laboratory constraints that can be placed on the same model. The overall conclusion is that LISA could place constraints on $f(R)$-gravity, which may be more powerful than those in the Solar System, but not as powerful as constraints from laboratory experiments. However, the LISA observations will probe a different energy scale, so these constraints will still be important, particularly if we regard $f(R)$ as an effective theory that could be different in different regimes. 

This paper is organised as follows. We begin with a review of the $f(R)$ field equations. In \secref{Lin} we derive the linearised equations and in \secref{Rad} we apply these to find wave solutions. These results can be used to study how gravitational radiation is modified for $f(R)$-gravity. They are largely known in the literature, but are worked out here {\it ab initio}; they are included as a compendium of useful results within a consistent system of notation. To be able to accurately model gravitational waveforms one needs to know how an object will inspiral. Accordingly, we derive an effective energy-momentum tensor for gravitational radiation in \secref{EM_tensor}, following the short-wavelength approximation of Isaacson~\cite{Isaacson1968, Isaacson1968a}. In \secref{Source} we look at the effects of introducing a source term and derive the weak-field metrics for a point source, a slowly rotating point source, and a uniform density sphere, recovering some results known for quadratic theories of gravity. These are used in \secref{Epicycle} to compute the frequencies of radial and vertical epicyclic oscillations about circular-equatorial orbits in the weak-field, slow-rotation metric, and hence to construct an estimate of the detectability of the $f(R)$ deviations in LISA EMRI observations. For comparison, in \secref{Tests}, we describe the constraints on $f(R)$-gravity that can be obtained from Solar System and laboratory tests. We conclude in \secref{f_Discuss} with a summary of our findings.

Throughout this work we will use the time-like sign convention of Landau and Lifshitz~\cite{Landau1975}:
\begin{enumerate}
\item The metric has signature $(+,-,-,-)$.
\item The Riemann tensor is defined as ${R^\mu}_{\nu\sigma\rho} = \partial_\sigma {\Gamma^\mu}_{\nu\rho} - \partial_\rho {\Gamma^\mu}_{\nu\sigma} + {\Gamma^\mu}_{\lambda\sigma}{\Gamma^\lambda}_{\rho\nu} - {\Gamma^\mu}_{\lambda\rho}{\Gamma^\lambda}_{\sigma\nu}$.
\item The Ricci tensor is defined as the contraction $R_{\mu\nu} = {R^\lambda}_{\mu\lambda\nu}$.
\end{enumerate}
Greek indices are used to represent spacetime indices $\mu = \{0,1,2,3\}$ (or $\mu = \{t,\widetilde{r},\theta,\phi\}$) and lowercase Latin indices are used for spatial indices $i = \{1,2,3\}$. Natural units with $c = 1$ will be used throughout, but factors of $G$ will be retained.

\chapter{Gravitation \& astrophysics}

\section{What goes up\ldots}

Gravity is one of the fundamental forces of nature; familiar as the force that keeps the Earth in orbit about the Sun and makes falling off a log so easy. \citet[book 3]{Newton1999} was the first to realise that gravitation could explain both apples falling from trees and the motion of astronomical bodies. In the \textit{Principia}, first published in 1687, he outlined a gravitational force that scaled as the inverse of square of the distance between centres of mass and was proportional to the product mass of the bodies. In modern notation the force is
\begin{equation}
F = \frac{G m_1 m_2}{r^2},
\end{equation}
for distance $r$, masses $m_1$ and $m_2$, and gravitational constant $G$. This theory has been hugely successful. Not only is it still taught in schools today, but it is also used for astronomical research. Newton's law of universal gravitation has proved accurate in describing orbital motions. However, there have been observations that did not fit its predictions.

In the early nineteenth century, the motion of Uranus was found to deviate from its expected trajectory. Rather than seeking to modify the theory, Couch Adams and Le Verrier calculated the properties of a perturbing object that could explain the motion. They predicted the existence of an unseen mass, a new planet; this was subsequently observed within a degree of Le Verrier's hypothesised position, and became known as Neptune.

Newtonian gravity survived the trial of Uranus' orbit, but it could not explain the perihelion precession of Mercury. Le Verrier first noticed the anomaly. A new inner planet was suggested, but this time it could not be found. What was needed was a modified theory of gravitation: the Newtonian theory is insufficient in the stronger gravity close to the Sun \citep[document 24]{Einstein1997}.

The new extended theory was General Relativity (GR), developed by Einstein in the 1910s \citep{Einstein1997}. This describes gravity as the effect of the curvature of spacetime, which is now a dynamical entity. Particles naturally travel along geodesics of spacetime, which may appeared curved; the curvature of spacetime itself is sourced from the energy-momentum in contains: matter tells spacetime how to curve, and spacetime tells matter how to move \citep[section 1.1]{Misner1973}. This is encapsulated within the Einstein field equations \citep[documents 22 and 25]{Einstein1997}
\begin{equation}
R_{\mu\nu} - \recip{2}R g_{\mu\nu} = \frac{8\pi G}{c^4}T_{\mu\nu},
\end{equation}
where $g_{\mu\nu}$ is the metric, $R_{\mu\nu}$ and $R$ and the Ricci tensor and scalar, $c$ is the speed of light, and $T_{\mu\nu}$ is the energy-momentum tensor. GR reduces to its Newtonian counter-part in the weak-field limit, or conversely, it extends Newtonian gravity to stronger gravitational fields.

Since its inception, GR has successfully passed every observational test \cite{Will1993, Will2006}. However, astronomers have not been idle, and the twentieth century has yielded further surprises.

Measurements of the rotation curves of galaxies do not match the expected profile calculated from their visible matter. Similarly, the velocity dispersion of galaxies in clusters is higher than would be expected. Gravitational lensing of galaxy clusters confirms that they gravitate more than expected from their apparent mass. This has been interpreted as motivation for introducing dark matter, a new component of the Universe that gravitates but does not interact with electromagnetic radiation. Dark matter has become central to our understanding of cosmology, it is needed to explain structure formation: without it we could not form galaxies from the small over-densities inferred from the homogeneity of the cosmic microwave background. Although we know the properties required of dark matter and we can estimation the required quantity, we do not have a definite candidate for a dark matter particle. Its true nature remains a mystery.

Observations of type IA supernovae have revealed that the Universe is not only expanding, but is accelerating. This acceleration has been attributed to the influence of dark energy. The nature of dark energy is even more mysterious than that of dark matter. The simplest explanation is to introduce a cosmological constant $\Lambda$, this modifies the Einstein field equations to become
\begin{equation}
R_{\mu\nu} - \recip{2}R g_{\mu\nu} + \Lambda g_{\mu\nu} = \frac{8\pi G}{c^4}T_{\mu\nu}.
\end{equation}
This model has been highly successful in explaining evolution of the Universe, but we still do not know if a cosmological constant is the true explanation and if so, why it has its particular value.

Despite its long history, we still do not know everything about gravity. There are still to discoveries to be made. Gravity is the weakest of the fundamental forces and so is difficult to study in the laboratory. Yet it dominates on astronomical scales; understanding gravity is crucial to understanding the cosmos. We have learnt much about the workings of the Universe through improving our understanding of gravity, and the motivation for developing new theories of gravitation has often come from astronomical observations. Gravitation and astrophysics are intimately linked.

This thesis is divided into two strands. The first is concerned with what we could learn about astrophysical systems from gravitational probes; the second is concerned with what we can learn about gravity from astronomical observations.  We shall consider strong-field tests and in particular gravitational waves. The former part concentrates on what we could hope to learn about massive black holes and their surrounding stellar environment from extreme-mass-ratio bursts. The latter looks at modifications to gravity in the metric $f(R)$ theory.

\section{Strong-field tests \& gravitational waves}

The deviations from Newtonian theory were first noted in the gravitational field close to the Sun, the strongest accessible in the Solar System. GR has now been tested in stronger fields \citep{Will2006}, but there are still more extreme systems to be explored. It is here that we would expect any deviations to manifest. We know that at least our understanding of GR in the strongest fields is incomplete, as black holes feature singularities at their centres, where the theory breaks down \citep[section 34.6]{Misner1973}. Even if we do not find any deviations from GR, it is still worthwhile to check its validity, if only as a matter of scientific principle.

\subsection{Field strength \& existing tests}

In order to parametrise the strength of gravity, \citet{Psaltis2008a} introduces two characteristic quantities: the dimensionless potential
\begin{equation}
\varepsilon = \frac{GM}{rc^2},
\end{equation}
and the dimensionful curvature
\begin{equation}
\xi = \frac{GM}{r^3c^2},
\end{equation}
where $M$ is the gravitating mass and $r$ a characteristic distance. These are larger for stronger fields. The potential ranges from $\varepsilon \simeq 0$ in weak fields to $\varepsilon = \order{1}$ at a black hole event horizon. It is useful in defining post-Newtonian expansions. The curvature $\xi$ approximates the form of the Ricci scalar, which is fundamental to GR. It is necessary to pick a particular reference scalar to define when the curvature becomes large; however, it is a useful gauge of the strength of a gravitational field in a geometric theory, because it is the lowest order measure that cannot be eliminated by a coordinate transformation \citep[chapter 7]{Hobson2006}.

Using these two parameters, we can map out the possible tests of GR\ldots

To probe the strongest fields, we need a way of probing the spacetime of compact objects like neutron stars and black holes\ldots 

\subsection{Gravitational radiation}

One particularly promising method of exploring strong-field regions would be to observe gravitational waves (GWs). These are predicted in any relativistic theory of gravity \citep{Schutz1984}; within GR they are tiny ripples in the spacetime metric. They are generated by systems with a time-varying mass quadrupole; significant gravitational radiation originates from regions where spacetime is highly dynamic and the objects are extremely relativistic. This is precisely the strong-field domain we are interested in investigating.

\subsubsection{Detection}

As yet no GWs have been directly detected, although their existence has been inferred from the loss of energy and angular momentum from binary pulsars\cite{Stairs2003}. There are a number of experiments designed to observe GWs. The Laser Interferometer Gravitational-wave Observatory (LIGO; \citealt{Abramovici1992}) and the European Gravitational Observatory's Virgo detector \citep{Acernese2008a}, which work in collaboration, are currently being upgraded to their advanced configurations and are expected to make the first detection shortly after recommencing operation around 2015 \citep{Harry2010,Accadia2011}.\footnote{An optimistic hope is to celebrate the centenary of Einstein's 1916 prediction of gravitation waves \citep[document 32]{Einstein1997} with the first direct detection.} These are ground-based interferometers that detect passing GWs by measuring the induced difference in the length of their two arms \citep{Pitkin2011}. They are sensitive to frequencies in the range $\sim10$--$10^4\units{Hz}$, with peak sensitivity at about $100\units{Hz}$. The LIGO and Virgo detectors are supported by GEO 600, a smaller interferometric experiment that incorporate prototype technologies \citep{Willke2002,Willke2006}. A further ground-based interferometer is under construction in Japan. The Kamioka Gravitational Wave Detector (KAGRA), formerly the the Large-scale Cryogenic Gravitational Wave Telescope (LCGT; \citealt{Kuroda1999,Kuroda2010}) will operate underground in the Kamioka mine. It lags several years behind the other detectors, but will employ more sophisticated noise-reduction techniques such as cryogenic cooling.

There is another contender for the first detection: pulsar timing arrays (PTAs) \citep{McWilliams2012,Sesana2012a}. These infer the presence of a GW from periodic delays in the arrival times of the highly regular millisecond pulses. They are sensitive to frequencies of $\sim10^{-9}$--$10^{-7}\units{Hz}$. An international collaboration of European, North American and Australian radio telescopes is already in possession of the necessary instruments to detect GWs \citep{Hobbs2010}.\footnote{The International Pulsar Timing Array (IPTA) consortium consists of the European Pulsar Timing Array, the (North American) NANOGrav and the (Australian) Parkes Pulsar Timing Array consortia.} The completion of the Square Kilometre Array (SKA; \citealt{Dewdney2009}) shall augment the search, greatly increasing sensitivity \citep{Kramer2004}.

Between the high frequency range of the ground-based detectors and the very low frequency range of pulsar timing, lies a band that could be accessible to space-based detectors.

While GWs are an exciting source of information, it will be beneficial to compare with results from other techniques, to maximise the data available for inferences, and to check models. For example, very long baseline interferometry (VLBI) may be used to image the vicinity of a BH's horizon, or X-ray observations could be used to investigate BH accretion discs\cite{Psaltis2008}.

\section{Black holes \& compact objects}

\section{Modified gravity}

%General relativity (GR) is a well tested theory of gravity~\cite{Will2006}; so far no evidence has been found that suggests it is not the correct classical theory of gravitation. However, there are many unanswered questions that remain regarding gravity which motivate the exploration of alternate theories: What are the true natures of dark matter and dark energy? How should we formulate a quantizable theory of gravity? What drove inflation in the early Universe?  Is GR the only theory that is consistent with current observations? Moreover, the majority of the tests that have been carried out to date have been in the weak-field, low-energy regime~\cite{Will2006, Will1993}: in the laboratory~\cite{Adelberger2009, Adelberger2003}, within the Solar System~\cite{Bertotti2003, Everitt2009} or using binary pulsars~\cite{Stairs2003}. It is not unreasonable to suppose that GR would begin to break down at higher energies.

%Over the coming decade, a new avenue for testing relativity will be opened up, through the detection of gravitational waves (GWs) using the existing ground-based GW detectors, the Laser Interferometer Gravitational-Wave Observatory (LIGO)~\cite{Abramovici1992, Abbott2009}, Virgo~\cite{Accadia2010} and GEO \cite{Willke2002, Abadie2010}, and the proposed space-based GW detector, the Laser Interferometer Space Antenna (LISA)~\cite{Bender1998, Danzmann2003}. These detectors will observe GWs generated during the inspiral and merger of binary systems comprising one or more black holes (BHs). The GWs are generated in the strong-field regime, while the components are highly relativistic and the spacetime is evolving dynamically: GW astronomy will open a new window into the strong-field regime of gravity, complementing traditional electromagnetic observations~\cite{Psaltis2008a}.  A comparison of the GWs observed from such systems with the predictions of GR will provide powerful tests of the theory in a region yet to be explored.

%%The radiation generated during the final merger and ring-down of two BHs will offer tests of GR in the highest energy and most dynamical sector, but it is thought that the most sensitive tests will come from LISA observations of extreme-mass-ratio inspirals (EMRIs)~\cite{Amaro-Seoane2007}. An EMRI involves the inspiral of a stellar-mass compact object, a white dwarf, neutron star or BH, into a massive BH in the centre of a galaxy. The mass of the compact object is typically $1$--$10M_{\odot}$, while the mass of the massive BH (for sources in the LISA band) will be $\sim10^5$--$10^7 M_\odot$, so the mass-ratio is of the order of $\sim10^{-7}$--$10^{-4}$. This extreme mass-ratio means that the inspiral proceeds slowly, and on short time-scales the compact object acts like a test particle moving in the background spacetime of the central BH. LISA will detect $\sim10^5$ cycles of gravitational radiation generated while the compact object is in the strong field of the spacetime, and this encodes a detailed map of the spacetime structure outside the central BH. This idea was first elucidated by Ryan~\cite{Ryan1995c, Ryan1997a}, who showed that, for an arbitrary stationary and axisymmetric spacetime in GR, the multipole moments of the spacetime enter at different orders in an expansion of the frequency of small vertical or radial oscillations of circular, equatorial orbits. As these frequencies are in principle observable in the GWs generated during an inspiral, it should be possible to measure the multipole moments from an EMRI observation and hence test whether the central object is a Kerr BH: according to the no-hair theorem, a Kerr BH is described completely by its mass $M$ and spin angular momentum $J$~\cite{Israel1967, Israel1968, Carter1971, Hawking1972, Robinson1975}, and its mass multipole $M_l$ ($M_0 \equiv M$) and mass-current multipole moments $S_l$ ($S_1 \equiv J$) are determined from these according to~\cite{Hansen1974}
%\begin{equation}
%M_l + iS_l = M \left(i\frac{J}{M}\right)^l .
%\end{equation}
%The multipole expansion is not a convenient way to characterise arbitrary spacetimes, since the Kerr metric itself requires an infinite number of multipoles to fully characterise. Subsequent authors have instead adopted the approach of considering bumpy BH spacetimes~\cite{Collins2004, Glampedakis2006a, Barack2007, Gair2008a}, which deviate from the Kerr metric by a small amount and depend on some parameter, $\epsilon$, such that $\epsilon = 0$ is precisely the Kerr solution. Relatively small perturbations to the Kerr solution can be detected in EMRI observations due to small differences in the precession frequencies that accumulate over the $100\,000$ waveform cycles that will be detected. There are also certain qualitative features that could be smoking-guns for a departure from the Kerr metric, such as ergodicity in the orbits~\cite{Gair2008a}, persistent resonances~\cite{Lukes-Gerakopoulos2010} or a shift in the frequency of plunge~\cite{Kesden2005, Gair2008a}.

%%The majority of the work to date has focused on spacetimes that are solutions in GR, but which deviate from the Kerr solution. However, if GR was not the correct theory of gravity, this could also lead to detectable signatures in the observed gravitational waves. Certain alternative theories of gravity, including $f(R)$, do admit the Kerr metric as a solution, since it has vanishing Ricci tensor, $R_{\mu\nu} = 0$~\cite{Psaltis2008, Yunes2011}. However, the Kerr metric need not be the expected end state of gravitational collapse~\cite{Barausse2008}. If a Kerr BH existed in an alternative theory, the geodesics would be the same, but the energy flux carried by the GWs could still be different, and so differences would show up in the rate of inspiral; although in many cases these differences do not appear at leading order. In most cases, however, either the Kerr metric is not admitted as a solution, or it is not the correct metric to describe collapsed objects~\cite{Yunes2011}. Waveform differences then show up as a result of the differences in the instantaneously-geodesic orbits of the compact object involved in the EMRI. Since the leading-order energy-momentum tensor of the GWs often takes the same form as in GR~\cite{Stein2011}, this is the primary effect and means the problem of testing alternative theories through EMRI observations is equivalent to the spacetime mapping programme within GR described previously.

%%As a consequence of the difficulties of solving for GW emission in alternative theories, work on testing alternative theories of gravity using LISA EMRIs has so far been restricted to a few cases. In Brans-Dicke gravity, in which the gravitational field is coupled to a scalar field, differences show up due to a modification to the inspiral rate that arises from dipole radiation of the scalar field~\cite{Berti2005}. Neutron star EMRIs are required since the dipole radiation depends on a sensitivity difference between the two objects, and the sensitivity is the same for all BHs. Lower mass central BHs provide the most powerful constraints, but a LISA observation of a neutron star EMRI into a $10^4 M_{\odot}$ BH could place constraints on the Brans-Dicke coupling parameter that are competitive with Solar System constraints~\cite{Berti2005}. In dynamical Chern-Simons modified gravity, the action is modified by a parity-violating correction, inspired by string theory~\cite{Alexander2008, Alexander2009a}. In this case, the BH solution differs from the Kerr solution at the fourth multipole, $l = 4$, but the energy-momentum tensor of gravitational radiation takes the same form as in GR~\cite{Sopuerta2009a}. LISA observations of EMRIs should place constraints on the Chern-Simons coupling parameter that are an order of magnitude better than will be possible from binary pulsar observations, although a full analysis accounting for parameter degeneracies has not yet been carried out~\cite{Sopuerta2009a}. 

\section{Conventions}

Throughout this work we will use the time-like sign convention of \citep{Landau1975}:
\begin{enumerate}
\item The metric has signature $(+,-,-,-)$.
\item The Riemann tensor is defined as ${R^\mu}_{\nu\sigma\rho} = \partial_\sigma {\Gamma^\mu}_{\nu\rho} - \partial_\rho {\Gamma^\mu}_{\nu\sigma} + {\Gamma^\mu}_{\lambda\sigma}{\Gamma^\lambda}_{\rho\nu} - {\Gamma^\mu}_{\lambda\rho}{\Gamma^\lambda}_{\sigma\nu}$.
\item The Ricci tensor is defined as the contraction $R_{\mu\nu} = {R^\lambda}_{\mu\lambda\nu}$.
\end{enumerate}
Greek indices are used to represent spacetime indices $\mu = \{0,1,2,3\}$ (or $\mu = \{t,\widetilde{r},\theta,\phi\}$) and lowercase Latin indices are used for spatial indices $i = \{1,2,3\}$. In general, factors of the speed of light $c$ and gravitational constant $G$ are retained, except for some sections where explicitly noted.

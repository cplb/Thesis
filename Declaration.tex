\chapter*{Declaration}
\markboth{Declaration}{}

This dissertation is the result of my own work and includes nothing which is the outcome of work done in collaboration, except where specifically indicated in the text. All concepts, results and data obtained by others are properly referenced. I have profited from discussion with and suggestions from a number of colleagues who are thanked in the \nameref{acknowledgements}. The use of ``we'' in the main text reflects my stylistic preference and nothing more.

This dissertation is not substantially the same as any that has been submitted for another degree, diploma or similar qualification. However, many of the results presented have been published as journal articles. These are:
\begin{quote}
C.P.L.B.\ \& Gair, J.R.; Gravitational wave energy spectrum of a parabolic encounter; \href{http://dx.doi.org/10.1103/PhysRevD.82.107501}{\it Physical Review D}; {\bf 82}(10):107501(4); November 2010; \linebreak \href{http://arxiv.org/abs/1010.3865}{\tt arXiv:1010.3865 [gr-qc]},
\end{quote}
which includes work included in \secref{Energy};
\begin{quote}
C.P.L.B.\ \& Gair, J.R.; Linearized $f(R)$ gravity: Gravitational radiation and Solar System tests; \href{http://dx.doi.org/10.1103/PhysRevD.83.104022}{\it Physical Review D}; {\bf 83}(10):104022(19); May 2011; \href{http://arxiv.org/abs/1104.0819}{\tt arXiv:1104.0819 [gr-qc]},
\end{quote}
which includes work included in chapters \ref{ch:f-R1} and \ref{ch:f-R2};
\begin{quote}
C.P.L.B.\ \& Gair, J.R.; Observing the Galaxy's massive black hole with gravitational wave bursts; \href{http://dx.doi.org/10.1093/mnras/sts360}{\it Monthly Notices of the Royal Astronomical Society}; {\bf 429}(1):589--612; February 2013; \href{http://arxiv.org/abs/1210.2778}{\tt arXiv:1210.2778 [astro-ph.HE]},
\end{quote}
which includes work included in chapters \ref{ch:waveforms} and \ref{ch:param}, and
\begin{quote}
C.P.L.B.\ \& Gair, J.R.; Extreme-mass-ratio-bursts from extragalactic sources; \href{http://dx.doi.org/10.1093/mnras/stt990}{\it Monthly Notices of the Royal Astronomical Society}; {\bf 433}(4):3572-3583; August 2013; \href{http://arxiv.org/abs/1306.0774}{\tt arXiv:1306.0774 [astro-ph.HE]},
\end{quote}
which includes work included in \secref{k-d} and \chapref{extragal}. Further to these, work from chapters \ref{ch:events} and \ref{ch:relax} has been accepted for publication as:
\begin{quote}
C.P.L.B.\ \& Gair, J.R.; Expectations for extreme-mass-ratio bursts from the Galactic Centre;{\it Monthly Notices of the Royal Astronomical Society}; accepted; \href{http://arxiv.org/abs/1307.7276}{\tt arXiv:1307.7276 [astro-ph.HE]}.
\end{quote}
Explicit citation to any of these indicates work done in collaboration. This is done in \secref{P-M}, where the original derivation of \eqnref{ell} is due to Jonathan Gair, and I subsequently verified this (correcting a minus sign and a factor of $\sqrt{2}$); in \secref{Epicycle}, where the calculation of the epicyclic frequencies was done by Jonathan Gair, generalising an earlier result of mine for the non-spinning metric, and I checked this (again fixing a minus sign), and in \secref{GW-f-R}, where the gravitational wave constraints were calculated using code developed by Jonathan Gair.

This dissertation contains less than $60000$ words.

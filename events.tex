\chapter{Event rates and expectations for Galactic bursts}\label{ch:events}

For EMRBs to be a valuable astronomical tool we require the bursts to loud enough to be detected, to contain sufficient information to improve our knowledge of their source systems and to have a sufficiently high event rate that we expect to observe them over a mission lifetime. We have previously found that bursts can satisfy the first two requirements: in \chapref{param} we found Galactic bursts can be informative if the periapse distance is $r\sub{p} \lesssim 10 r\sub{g}$ and in \chapref{extragal} we found that the most promising extragalactic bursts could be informative if $r\sub{p} \lesssim 8 r\sub{g}$. We address the final requirement in this chapter.

\section{Calculating event rates}\label{sec:Rates}

Having determined how to generate a waveform and extract the information from it, we must now consider how likely it is that such a waveform would be observed. We wish to calculate the event rate for EMRBs, the probability that there is an encounter between a CO, on an orbit described by eccentricity $e$ and periapse radius $r\sub{p}$, and the MBH. To do so we must build a model to describe the distribution of COs about the MBH. The number density of stars in the six-dimensional phase space of position and velocity is described by the distribution function (DF) $f$ \citep[section 4.1]{Binney2008}. We introduce approximate forms for the DF appropriate for describing the Galactic core in \secref{DF}. These are calibrated using the simulations of \citet{Alexander2009}, the parameters of which, together with the others used to describe the Galactic core, are given in \secref{GC-Param}. Having set the distribution of COs, we explain how to convert this to an event rate in \secref{e-rp}. In \eqnref{Gamma} we give an expression that relates the event rate for an orbit $\Gamma(e,\,r\sub{p})$ and the DF. There is one final consideration before we can calculate the total event rate: that there is an inner periapsis below which orbits become depopulated. This is carefully explained in \secref{inner-cut}. We consider tidal disruption and collisions, which we assume truncate the DF at a finite periapsis so that the event rate inside these cut-offs is zero. We also consider GW inspiral, which we assume alters the event rate by modifying the distribution of COs away from its relaxed state. With these inner cut-offs established, we have completely defined the event rate distribution. This can then give the probability of an EMRB and the total event rates, which are presented in \secref{no-events}.

\subsection{The distribution function}\label{sec:DF}

Following the work of \citet{Bahcall1976, Bahcall1977}, we assume the DF within the Galactic core is only a function of the orbital energy \citep{Shapiro1978}. The energy per unit mass of the orbit is
\begin{equation}
\mathcal{E} = \frac{v^2}{2} - \frac{GM_\bullet}{r},
\end{equation}
where $v$ is orbital velocity. The number of stars is
\begin{equation}
N = \int \dd^3r \int \dd^3v f(\mathcal{E}).
\end{equation}
Close to the centre of the Galactic core, the dynamics are dominated by the influence of the MBH as it is significantly more massive than the surrounding stars. Its radius of influence is
\begin{equation}
r\sub{c} = \frac{GM_\bullet}{\sigma^2},
\label{eq:r_c}
\end{equation}
where $\sigma^2$ is the line-of-sight velocity dispersion \citep{Frank1976}. We assume that the mass of stars enclosed within $r\sub{c}$ is greater than the $M_\bullet$, which, in turn, is much greater than the mass of a typical star $M_\star$ \citep{Bahcall1976}. We define a reference number density $n_\star$ from the enclosed mass $m_\ast(r)$ such that
\begin{equation}
m_\star(r\sub{c}) = \frac{4\pi r\sub{c}^3}{3}n_\star M_\star.
\end{equation}
Within the core, the DF can be calculated using the approximation of the Fokker--Planck formalism \citep[section 7.4]{Binney2008}. The population of bound stars is evolved numerically until a steady state is reached, whilst the unbound stars form a reservoir with an assumed Maxwellian distribution. Denoting a species of star by its mass $M$, the unbound DF is
\begin{equation}
f_M(\mathcal{E}) = \frac{C_M n_\star}{(2\pi\sigma_M^2)^{3/2}} \exp\left(-\frac{\mathcal{E}}{\sigma_M^2}\right),\quad\mathcal{E} > 0,
\label{eq:Unbound_DF}
\end{equation}
where $C_M$ is a normalisation constant.\footnote{$C_M$ determines the population ratios of species $M$ far from the black hole \citep{Alexander2009}.} If different stellar species are in equipartition, as assumed by \citet{Bahcall1976, Bahcall1977}, we expect
\begin{equation}
M \sigma_M^2 = M_\star \sigma_\star^2.
\end{equation}
However, if the unbound stellar population has reached equilibrium by violent relaxation, all mass groups are expected to have similar dispersions:
\begin{equation}
\sigma_M = \sigma_\star = \sigma,
\end{equation}
and we have equipartition of energy per unit mass \citep{Lynden-Bell1967}. This is assumed here following \citet{Alexander2009} and \citet{O'Leary2009}. The steady-state DF is largely insensitive to this choice \citep{Bahcall1977, Alexander2009}.

For bound orbits, the DF can be approximated as a power law \citep{Peebles1972}
\begin{equation}
f_M(\mathcal{E}) = \frac{k_M n_\star}{(2\pi\sigma^2)^{3/2}}\left(-\frac{\mathcal{E}}{\sigma^2}\right)^{p_M},\quad\mathcal{E} < 0.
\label{eq:Bound_DF}
\end{equation}
The exponent $p_M$ varies depending upon the mass of the object, determining mass segregation. For a system with a single mass component $p = 1/4$ \citep{Bahcall1976, Young1977}. The normalisation constant $k_M$ reflects the relative abundances of the different species.\footnote{For a single mass population ($p = 1/4$) $k = 2 C$ gives a fit correct to within a factor of two \citep{Bahcall1976,Keshet2009}, we assume this holds for the dominant species of stars as, although it changes slightly with $p$, variation is small compared to errors introduced by fitting a simple power law \citep{Hopman2006, Alexander2009}.}

These cusp profiles should exist if the system has had sufficient time to become gravitationally relaxed. There is current debate about whether this may be the case, both for the Galactic Centre and galaxies in general. This is discussed further in \apref{tauGC}. For concreteness, we assume a cusp has formed. If a cusp has not formed, we expect there to be a shallower core profile, with fewer objects passing close to the MBH. Our results are therefore an upper bound on possible event rates \citep{Merritt2010a,Gualandris2012}. 

\subsection{Model parameters}\label{sec:GC-Param}

We use the Fokker--Planck model of \citet{Hopman2006, Hopman2006a} and \citet{Alexander2009}. This includes four stellar species: MS stars, WDs, NSs and stellar mass BHs. Their properties are summarised in \tabref{HA}. The behaviour of the Fokker--Planck model has been verified by $N$-body simulations \citep{Baumgardt2004,Preto2010}.
%\begin{table}
%\begin{minipage}{\columnwidth}
% \centering
%  \caption{Stellar model parameters for the Galactic core using the results of \citet{Alexander2009}. The main sequence star is used as a reference for the normalisation constants. The number fractions for unbound stars are estimates corresponding to a model of continuous star formation \citep{Alexander2005}; \citet{O'Leary2009} arrive at the same proportions.\label{tab:HA}}
%  \begin{tabular}{@{} l D{.}{.}{2.1} D{.}{.}{1.3} D{.}{.}{1.1} D{.}{.}{1.3} @{}}
%  \hline
%   Star & \multicolumn{1}{c}{$M/M_\odot$} & \multicolumn{1}{c}{$C_M/C_\star$} & \multicolumn{1}{c}{$p_M$} & \multicolumn{1}{c}{$k_M/k_\star$\footnote{\citet*{Toonen2009}}} \\
% \hline
% MS & 1.0 & 1 & -0.1 & 1 \\
% WD & 0.6 & 0.1 & -0.1 & 0.09 \\
% NS & 1.4 & 0.01 & 0.0 & 0.01  \\
% BH & 10 & 0.001 & 0.5 & 0.008 \\
%\hline
%\end{tabular}
%\end{minipage}
%\end{table}
The steeper power law for BHs means they segregate about the MBH.\footnote{Extrapolating, they would dominate in place of MS stars for radii $r < 10^{-4}r\sub{c}$.}

Binaries may form in the Galactic core, encouraged by its high stellar density \citep{O'Leary2009}. However the binary fraction is still expected to be small \citep{Hopman2009}. Binaries are also disrupted by the MBH for periapses smaller than
\begin{equation}
r\sub{B} \simeq \left(\frac{M_\bullet}{M_1 + M_2}\right)^{1/3}a\sub{B},
\end{equation}
where $M_1$ and $M_2$ are the masses of the binary's components, and $a\sub{B}$ is the binary's semi-major axis, cf.\ \eqnref{Tidal} below. Thus, we ignore the possible presence of binaries.

We assume $M_\bullet = (4.31 \pm 0.36) \times 10^6 M_\odot$ \citep{Gillessen2009} and $\sigma = (103 \pm 20)\units{km\,s^{-1}}$ \citep{Tremaine2002}. This gives a core radius of $r\sub{c} = (1.7 \pm 0.7)\units{pc}$. Using the results of \citet{Ghez2008} we would expect the total mass of stars in the core to be $m_\star(r\sub{c}) = 6.4 \times 10^6 M_\odot$, which is within $5\%$ of the value obtained similarly from \citet{Genzel2003}. This gives a reference stellar density of $n_\star = 2.8 \times 10^5\units{pc^{-3}}$.

\subsubsection{Relaxation time-scale}\label{sec:Relax}

The motion of a star is determined not only by the dominant influence of the central MBH, but also by the other stars. The gravitational potential of the stars may be split into two components: a smooth background representing the average distribution of stars, and statistical fluctuations from random deviations in the stellar distribution because of individual stellar motions. The former only contributes to the stars' orbits: we neglect this since we are more interested in the influence of the MBH. The latter may be approximated as a series of two-body encounters. These lead to scattering, in a manner much like Brownian motion \citep{Bekenstein1992,Maoz1993,Nelson1999}.

The two-body interactions mostly lead to small deflections. Over time, these may accumulate into a significant change in the dynamics. The relaxation time-scale characterises the time taken for this to happen \citep[section 1.2.1]{Binney2008}. It therefore quantifies the time over which an orbit may be repopulated by scattering.

There are a variety of different methods used to define a relaxation time-scale. We follow the classic treatment of \citet[chapter 2]{Chandrasekhar1960}, adapting from a Maxwellian distribution of velocities to one derived from the DFs \eqnref{Unbound_DF} and \eqnref{Bound_DF}; this makes the model self-consistent. The derivation of the relaxation time-scale is found in \chapref{relax}, since it is too long to include here. An average time-scale for the entire system $\overline{\tau\sub{R}}$ is defined in \eqnref{system-relax}, and an average for an orbit $\left\langle\tau\sub{R}\right\rangle$ is defined in \eqnref{orbital-relax}. 

Two-body interactions lead to diffusion in both energy and angular momentum. When considering a single (bound) orbit, over a relaxation time-scale the energy changes by order of itself while the angular momentum changes by the angular momentum of a circular orbit with that energy $\mathcal{J}\sub{circ}(\mathcal{E})$ \citep{Lightman1977, Rauch1996, Hopman2005, Madigan2011}:\footnote{$\mathcal{J}\sub{circ}(\mathcal{E})$ is the maximum value for orbits of that energy.}
\begin{equation}
\left(\dfrac{\Delta\mathcal{E}}{\mathcal{E}}\right)^{2} \approx \left[\dfrac{\Delta \mathcal{J}}{\mathcal{J}\sub{circ}(\mathcal{E})}\right]^{2} \approx \dfrac{t}{\tau\sub{R}}.
\label{eq:diffuse-relax}
\end{equation}
We may define another angular momentum relaxation time-scale as the time taken for the angular momentum to change by order of itself \citep{Merritt2011}
\begin{align}
\tau_\mathcal{J} = {} & \left[\dfrac{\mathcal{J}}{\mathcal{J}\sub{circ}(\mathcal{E})}\right]^2\tau\sub{R} = \left(1 - e^2\right) \tau\sub{R}.
\label{eq:J-time}
\end{align}
This can be much shorter than the energy relaxation time-scale: diffusion in angular momentum can proceed more rapidly than diffusion in energy.

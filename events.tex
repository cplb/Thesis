\chapter{Event rates and expectations for Galactic bursts}\label{ch:events}

\subsubsection{Relaxation time-scale}\label{sec:Relax}

The motion of a star is determined not only by the dominant influence of the central MBH, but also by the other stars. The gravitational potential of the stars may be split into two components: a smooth background representing the average distribution of stars, and statistical fluctuations from random deviations in the stellar distribution because of individual stellar motions. The former only contributes to the stars' orbits: we neglect this since we are more interested in the influence of the MBH. The latter may be approximated as a series of two-body encounters. These lead to scattering, in a manner much like Brownian motion \citep{Bekenstein1992,Maoz1993,Nelson1999}.

The two-body interactions mostly lead to small deflections. Over time, these may accumulate into a significant change in the dynamics. The relaxation time-scale characterises the time taken for this to happen \citep[section 1.2.1]{Binney2008}. It therefore quantifies the time over which an orbit may be repopulated by scattering.

There are a variety of different methods used to define a relaxation time-scale. We follow the classic treatment of \citet[chapter 2]{Chandrasekhar1960}, adapting from a Maxwellian distribution of velocities to one derived from the DFs \eqnref{Unbound_DF} and \eqnref{Bound_DF}; this makes the model self-consistent. The derivation of the relaxation time-scale is found in \chapref{relax}, since it is too long to include here. An average time-scale for the entire system $\overline{\tau\sub{R}}$ is defined in \eqnref{system-relax}, and an average for an orbit $\left\langle\tau\sub{R}\right\rangle$ is defined in \eqnref{orbital-relax}. 

Two-body interactions lead to diffusion in both energy and angular momentum. When considering a single (bound) orbit, over a relaxation time-scale the energy changes by order of itself while the angular momentum changes by the angular momentum of a circular orbit with that energy $\mathcal{J}\sub{circ}(\mathcal{E})$ \citep{Lightman1977, Rauch1996, Hopman2005, Madigan2011}:\footnote{$\mathcal{J}\sub{circ}(\mathcal{E})$ is the maximum value for orbits of that energy.}
\begin{equation}
\left(\frac{\Delta\mathcal{E}}{\mathcal{E}}\right)^{2} \approx \left[\frac{\Delta \mathcal{J}}{\mathcal{J}\sub{circ}(\mathcal{E})}\right]^{2} \approx \frac{t}{\tau\sub{R}}.
\label{eq:diffuse-relax}
\end{equation}
We may define another angular momentum relaxation time-scale as the time taken for the angular momentum to change by order of itself \citep{Merritt2011}
\begin{align}
\tau_\mathcal{J} = {} & \left[\frac{\mathcal{J}}{\mathcal{J}\sub{circ}(\mathcal{E})}\right]^2\tau\sub{R} = \left(1 - e^2\right) \tau\sub{R}.
\label{eq:J-time}
\end{align}
This can be much shorter than the energy relaxation time-scale: diffusion in angular momentum can proceed more rapidly than diffusion in energy.

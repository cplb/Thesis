\chapter{Black holes and revelations}\label{ch:All-good-things}

\section{Astrophysics and gravitation}

\subsection{Extreme-mass-ratio bursts and massive black holes}

\subsetion{Metric $f(R)$-gravity and gravitational radiation}

\subsection{Extreme-mass-ratio inspirals and transient resonances}

\section{The horizon}

\section{\ldots must come down}

We begin by studying how extreme-mass-ratio bursts (EMRBs), a particular class of gravitational wave signals, could be used to determine the properties of the massive black holes found in the centres of galaxies. EMRBs could be detectable with a space-borne detector from systems containing a $M_\bullet \sim 10^6$--$10^7 M_\odot$ black hole out to a distance of $\sim 100\units{Mpc}$. They could provide insightful constraints on the black holes' parameters, principally mass and spin, which could elucidate the formation history of the black holes and, by association, their host galaxies. In the case of our Galaxy, EMRBs need to originate from orbits with periapse radii of $r\sub{p} \lesssim 16 r\sub{g}$, where $r\sub{g} = GM_\bullet/c^2$ is a gravitational radius, to be informative; to yield results better than currently available, the orbits need to have $r\sub{p} \lesssim 10 r\sub{g}$. For the most promising extragalactic candidates, such as M32, the periapsis must be $r\sub{p} \lesssim 8 r\sub{g}$ for EMRBs to provide useful measurements.




In \partref{astro} we studied EMRBs and how they can be used to learn about MBHs, as well as investigating the properties of the stellar core that surrounds an MBH. There are still many uncertainties about Galactic cores, their stellar distributions, MBHs and the processes that couple them.

EMRBs could be a useful medium for learning about MBHs. Bursts are detectable from systems containing MBHs of masses $\sim 10^6$--$10^7 M_\odot$. They are of greatest utility for investigating the Galactic MBH. We have found that with a spaced-based mission comparable to LISA, we could expect to gain $2.2\units{nats}$ of information about the logarithm of the mass, $3.0\units{nats}$ about the spin magnitude, $2.8\units{nats}$ about the cosine of the polar angle for the spin axis and $4.2\units{nats}$ for the azimuthal angle.

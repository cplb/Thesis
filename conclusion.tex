In \partref{astro} we studied EMRBs and how they can be used to learn about MBHs, as well as investigating the properties of the stellar core that surrounds an MBH. There are still many uncertainties about Galactic cores, their stellar distributions, MBHs and the processes that couple them.

EMRBs could be a useful medium for learning about MBHs. Bursts are detectable from systems containing MBHs of masses $\sim 10^6$--$10^7 M_\odot$. They are of greatest utility for investigating the Galactic MBH. We have found that with a spaced-based mission comparable to LISA, we could expect to gain $2.2\units{nats}$ of information about the logarithm of the mass, $3.0\units{nats}$ about the spin magnitude, $2.8\units{nats}$ about the cosine of the polar angle for the spin axis and $4.2\units{nats}$ for the azimuthal angle.

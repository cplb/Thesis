\chapter{Black holes and revelations}\label{ch:All-good-things}

\section{Astrophysics and gravitation}

In \chapref{Let-there-be-gravity} we explained the central role of gravity in astrophysics, how advancements in our understanding of gravity are driven by astronomical measurements and how gravitational interactions can give insight into astrophysical systems. We argued that the currently unexplored strong-field regime may be a source of interesting discoveries.

Over the course of this dissertation we have studied what could be learnt about both astrophysical systems and gravity. We have concentrated on using GWs as a tool, specialising in EMR sources. We have investigated using EMRBs to discover the properties of MBHs (\secref{Review-EMRBs}); calculated the observable differences between GR and metric $f(R)$-gravity (\secref{Review-f-R}), and looked at the effect of transient resonances on the evolution of EMRIs (\secref{Review-resonances}). We have seen that strong-field tests can be informative, but that there are still things to be learnt from weak fields too.

\subsection{Extreme-mass-ratio bursts and massive black holes}\label{sec:Review-EMRBs}

In \partref{astro} we studied EMRBs and how they can be used to learn about MBHs, as well as investigating the properties of the NSC that surrounds an MBH. There are still many uncertainties about galactic centres, their stellar distributions, MBHs and the processes that couple them. If bursts could provide insightful constraints on the MBHs' parameters, principally mass and spin, they could elucidate the formation history of the MBHs and, by association, their host galaxies.

We began in \chapref{waveforms} by constructing EMRB waveforms. This was done using the semirelativistic NK approximation. To verify the accuracy of the waveforms, we derived the GW energy spectrum for a Keplerian parabolic orbit and confirmed that this matched the NK spectra for orbits with a large periapsis. We also compared the total energy flux with that derived from BH perturbation theory. From these comparisons, we estimated that the typical error in the NK waveforms was the order of a few percent.

In \chapref{param} we investigated the potential for EMRBs to be used for inferring the properties of the Galactic MBH. We first determined that Galactic bursts are loud enough to be a credible source. The SNR, assuming the LISA noise curve, can be well approximated by a simple power law
\begin{equation}
\log\rho \simeq -2.7\log\left(\dfrac{r\sub{p}}{r\sub{g}}\right) + \log\left(\dfrac{\mu}{M_\odot}\right) + 4.9;
\end{equation}
assuming a $\mu = 10 M_\odot$ CO, bursts would be detectable for periapse distances $r\sub{p} \lesssim 65 r\sub{g}$. Using eLISA the SNR is reduced; it can be approximated by 
\begin{equation}
\log\rho \simeq -2.9\log\left(\dfrac{r\sub{p}}{r\sub{g}}\right) + \log\left(\dfrac{\mu}{M_\odot}\right) + 3.9,
\end{equation}
and bursts, assuming $\mu = 10 M_\odot$, are detectable for $r\sub{p} \lesssim 21 r\sub{g}$.

Having determined that bursts could be detected, we investigated their information content. MCMC simulations were used to map out the posterior distributions calculated for a selection of bursts assuming detection with LISA. Characterising the distributions by their widths, we found that bursts could be informative if $r\sub{p} \lesssim 16 r\sub{g}$; to yield results better than those currently available, the orbits need to have $r\sub{p} \lesssim 10 r\sub{g}$.

Encouraged by results for Galactic bursts, in \chapref{extragal} we considered the possibility of detecting bursts from extragalactic sources. Using LISA, bursts are detectable from systems containing MBHs of masses $M_\bullet \sim 10^6$--$10^7 M_\odot$ out to distances of $R \sim 100\units{Mpc}$. This range encompasses a number of potential sources. For eLISA the range is reduced by approximately an order of magnitude, but this still leaves several candidate systems. M32 is the most promising extragalactic source.

We considered the information content of extragalactic bursts, using M32, NGC 4395 and NGC 4945 as example systems. While extragalactic bursts are not as informative as their Galactic counterparts, we found that if $r\sub{p} \lesssim 8 r\sub{g}$ the bursts provide useful measurements.

To conclude our analysis of EMRBs, we calculated the event rate for Galactic bursts, the most likely source. Building a model for the Galactic centre, we estimated that there could be $\Gamma \approx 0.8\units{yr^{-1}}$ detectable bursts with LISA. As a consequence of mass segregation, stellar-mass BHs provide the majority of events, giving a rate of $\Gamma\sub{BH} \approx 0.6\units{yr^{-1}}$. For eLISA the rates are reduced to $\Gamma \approx 0.5\units{yr^{-1}}$ total and $\Gamma\sub{BH} \approx 0.3\units{yr^{-1}}$ for stellar-mass BHs. The event rate is not high, but still enough to make EMRBs a credible source over a mission lifetime.

Using the event rate model, we constructed expectations for the amount of information that could be gained about the Galactic MBH. To quantify this, we used the information entropy of the burst posterior distributions relative to the priors. We found that with a space-borne mission comparable to LISA, we could expect to gain $2.2\units{nats}$ of information about the logarithm of the MBH mass, $3.0\units{nats}$ about the spin magnitude, $2.8\units{nats}$ about the cosine of the polar angle for the spin axis and $4.2\units{nats}$ for the azimuthal angle. EMRBs could be a useful medium for learning about MBHs; they are of greatest utility for investigating the Galactic MBH.

In building the event rate model it was necessary to incorporate the physics of the NSC that surrounds the MBH. The distribution of COs relaxes through two-body encounters. We calculated appropriate relaxation times for the NSC in \chapref{relax}. These are comparable to the canonical values calculated using Maxwellian velocity distributions. The relaxation time-scales give some insight into the population of COs close to the MBH, highlighting that the assumed formation of a cusp is not certain. The analysis of the GC emphasises how understanding Newtonian interactions is still an important problem in astrophysics.

\subsection{Metric $f(R)$-gravity and gravitational radiation}\label{sec:Review-f-R}

In \partref{grav} we moved on to considering our current understanding of gravity: only if we have an accurate comprehension of gravitational interactions can we use them to make inferences about astrophysical systems.

We began by formulating the consequences of gravity being described by the metric $f(R)$ theory. This alternative theory was introduced in \chapref{f-R1}. It shares many of the desirable properties of GR, but includes extra freedom through the arbitrary choice of the function $f(R)$. This gives $f(R)$-gravity the potential to match a range of phenomena. We adopted a functional form
\begin{equation}
f(R) = a_0 + R + \frac{a_2}{2}R^2 + \frac{a_3}{6}R^3 + \ldots
\end{equation}
with $a_0 = 0$ to permit Minkowski spacetime as a vacuum solution.

Gravitational radiation is modified in $f(R)$-gravity. As in GR, there are two transverse GW polarizations which satisfy the vacuum wave equation
\begin{equation}
\Box \overline{h}_{\mu\nu} = 0,
\end{equation}
but instead of being the trace-reversed metric perturbation, the quantity $\overline{h}_{\mu\nu}$ also depends upon the (first-order) Ricci scalar according to
\begin{equation}
\overline{h}_{\mu\nu} = h_{\mu\nu} - \left(a_2 R^{(1)} + \dfrac{h}{2}\right)\eta_{\mu\nu}.
\end{equation}
Together with the transverse modes, there is an additional scalar mode governed by vacuum wave equation
\begin{equation}
\Box R^{(1)} - \recip{3 a_2} R^{(1)} = 0.
\end{equation}
This is qualitatively different from in GR; detection of the scalar mode would be decisive evidence against GR.

The form of the effective energy--momentum tensor is also modified. Using the short-wavelength approximate, we derived that the effective tensor is
\begin{equation}
t_{\mu\nu} = \recip{32\pi G}\left\langle \partial_\mu\overline{h}_{\sigma\rho}\partial_\nu\overline{h}^{\rho\sigma} + 6a_2^2\partial_\mu R^{(1)}\partial_\nu R^{(1)} \right\rangle.
\end{equation}
Additional energy--momentum is carried by the scalar mode (if it is propagating). The effective energy--momentum tensor correctly reduces to the GR form in the limit of $a_2 \rightarrow 0$.

In \chapref{f-R2} we considered the observational constraints that could be used to determine the form of $f(R)$. We derived the weak-field metric for a point mass. Since the BH solutions are the same as in GR, we must use measurements of extended bodies. Consequently, EMR measurements of the structure of the spacetime of MBHs are not of use here, although the form of the radiation could be informative. We considered the classic tests of light-deflection and planetary precession. The former is identical to that in GR to PN order; hence it is fully consistent with current observations. The perihelion precession of Mercury imposes the bound $|a_2| \lesssim 1.2 \times 10^{18}\units{m^2}$. However, a tighter bound can be achieved using laboratory fifth-force tests, $|a_2| \lesssim 2 \times 10^{-9}\units{m^2}$.

These weak-field tests make it unlikely that we could hope to see deviations in the strong field, assuming that $f(R)$-gravity is a universally applicable theory. This need not be the case if it only serves as an effective theory. Additionally, there could be a screening effect, such as the chameleon mechanism, that suppresses the deviations from GR in the vicinity of matter. Therefore, strong-field tests are needed to prove decisively that GR is the correct theory of gravity.

\subsection{Extreme-mass-ratio inspirals and transient resonances}\label{sec:Review-resonances}

To conclude our study of gravitation, in \chapref{resonances} we investigated the effect of passing through a transient resonance on an EMRI evolution. Accurate waveforms are needed for either astronomical measurements or searching for deviations from GR, hence it is important to understand the impact of resonances. The behaviour when evolving through resonances is governed by the gravitational self-force; calculation of the self-force is an active area of research.

We mapped out the location of resonances, providing an approximate fit for the semilatus rectum for a particular resonance as a function of MBH spin, orbital inclination and eccentricity. We then used matched asymptotic expansions to estimate the change in the orbital parameters across a resonance. The change for $\mathcal{I}_a = \{E,L_z,Q\}$ is given by
\begin{equation}
\Delta \mathcal{I}_a = \eta\sum_{s\,\neq\,0}F_{a,\,s}^{(1)}(\boldsymbol{\mathcal{I}}_\star)\tau_{\mathrm{res},\,s}\exp\left\{i\left[s \widehat{\kappa}_0 +\sgn\left(s\dot{\Omega}\right)\dfrac{\pi}{4}\right]\right\};
\end{equation}
as expected, this scales with the mass ratio $\eta$ and depends upon the magnitude of the self-force on resonance $F_{a,\,s}^{(1)}(\boldsymbol{\mathcal{I}}_\star)$, the resonance time-scale $\tau_{\mathrm{res},\,s}$ and a term that oscillates with the orbital phase on resonance $\widehat{\kappa}_0$ such that the average is zero.

\section{The horizon}

There remain open questions surrounding the topics covered in this work. Many of these should be answered as we begin to amass observational data from strong-field tests, although we expect new questions to arise too. There is a multitude of problems that must be addressed before GW astronomy can become a mature science. Within the confines of the work considered here, there are some (smaller) problems that remain to be addressed; these could be fruitful subjects for further study.

With regards to EMRBs, no work has yet be done on formulating a detection algorithm. While we have demonstrated that bursts can have high SNRs, we have not shown how they can be identified from a data stream. Since they have a well defined frequency spectrum, it should be possible to use matched filtering \citep[cf.][]{Feroz2010}; however, a practical implementation of this has yet to be attempted.

An interesting extension to the work on $f(R)$-gravity is to consider the case when the constant term in the function $f(R)$, $a_0$, is non-zero. In this case we study perturbations with respect to (anti-)de Sitter space. This is relevant as the current $\Lambda$CDM paradigm indicates that we live in a universe with a positive cosmological constant \citep{Komatsu2011,Hinshaw2012,Ade2013b}. Such a study naturally complements an investigation into the effects of background curvature on propagation \citep{Yang2011}. Furthermore, \citet{Sotiriou2011} assumed asymptotic flatness when proving that the BHs of $f(R)$-gravity must be the Kerr solutions; relaxing this assumption has yet to be conclusively investigated.

Only preliminary work has been done on the influence of transient resonances on EMRI waveform analysis. The next step is to consider the waveform dephasing relative to an adiabatic evolution. A more sophisticated analysis of the impact of resonances should incorporate investigation of their influence on parameter estimation. If the effect of passing through a resonance is small, they could be neglected for the purposes of detection. However, there is the possibility that doing so when inferring parameters leads to a systematic error. If the effect of a resonance is large, then there could be significant differences between parameters calculated assuming an adiabatic evolution and the true parameters. Any calculations require a self-force model, hence it is necessary to build an adequate description for this purpose.

\section{\ldots must come down}

Strong-field tests can provide new information regarding astrophysical objects and the nature of gravitation. Following the first direct detection of GWs, GW astronomy shall give us a new tool for examining the Universe and a novel means of measuring the properties of COs. This should result in many exciting discoveries. However, weak-field tests are still of great use, and there are aspects of systems whose properties are described by Newtonian dynamics which are still to be fully understood. Both the gravity of Newton and the gravity of Einstein are of central importance to astrophysics; whether a further theory is needed remains to be discovered.
\section{Discussion}\label{sec:End}

We have outlined an approximate method of generating gravitational waveforms for EMRBs. This assumes that the orbits are parabolic and employs a numerical kludge approximation. The two coordinate schemes for an NK presented here yield almost indistinguishable results. We conclude that either is a valid choice for this purpose. There may be differences when the spin is large and the periapse is small: $\sim 10\%$ for $r\sub{p} \simeq 4 r\sub{g}$, $\sim 20\%$ for $r\sub{p} \simeq 2 r\sub{g}$.

The waveforms created appear to be consistent with results obtained using Peters and Mathews waveforms for large periapses, indicating that they have the correct weak-field form. The NK approach should be superior to that of Peters and Mathews in the strong-field regime as it uses the exact geodesics of the Kerr spacetime. Comparisons with energy fluxes from black hole perturbation theory indicate that typical waveform accuracy may be of order $5\%$, but this is worse for orbits with small periapses, and may be $\sim 20\%$. These errors are greater than the differences resulting from the use of the alternative coordinate systems.

The signal-to-noise ratio of bursts is well correlated with the periapsis. For bursts from from the GC the SNR (per unit mass) may be reasonably described as having a power-law dependence of
\begin{equation}
\log\left(\hat{\rho}\right) \simeq -2.7\log\left(\frac{r\sub{p}}{r\sub{g}}\right) + 4.9,
\end{equation}
except for the closest orbits ($r\sub{p} \lesssim 7 r\sub{g}$). Signals should be detectable for a $1 M_\odot$ ($10 M_\odot$) object if the periapse is $r\sub{p} < 27 r\sub{g}$ ($r\sub{p} < 65 r\sub{g}$), corresponding to a physical scale of $1.7 \times 10^{11}\units{m}$ ($4.1 \times 10^{11}\units{m}$) or $5.6 \times 10^{-6}\units{pc}$ ($1.3 \times 10^{-5}\units{pc}$).

We used MCMC results as a robust measure of parameter estimation accuracy. Potentially, it is possible to determine very precisely the key parameters defining the Galaxy's MBH's mass and spin, if the periapsis is sufficiently small. From our investigation it appears that we can achieve good results from a single EMRB with periapsis of $r\sub{p} \simeq 10 r\sub{g}$ for a $10 M_\odot$ CO. This translates to a distance of $6 \times 10^{10}\units{m}$ or $2 \times 10^{-6}\units{pc}$. Orbits closer than this would be even better, and place stricter constraints. The best orbits yield uncertainties of almost one part in $10^5$ for the MBH mass and spin, far exceeding existing techniques. Conversely, orbits with $r\sub{p} \gtrsim 20 r\sub{g}$ are unlikely to provide any useful information.

To estimate the event rate, we constructed a model of dynamical processes in the GC. Parametrizing orbits by their periapsis of eccentricity, we imposed a number of cuts to account for tidal disruptions, collisions and the effects of gravitational wave inspiral. Whilst results are not as accurate as if obtained using $N$-body simulations, the should be a reasonable and relatively inexpensive approximation. We calculate an expected event rate of $\sim 1.7$ per two year mission. Stellar mass BHs are the most likely source, although there is also a non-negligble contribution from NSs.

While we have only considered bursts from our own galaxy in detail, it should be possible to observe bursts from other nearby galaxies if their MBH is of the appropriate mass. The SNR of EMRBs obeys a number of scaling relations that allow us to check whether an MBH could produce detectable bursts. M32 is the best extragalactic candidate. However, even in this case, the region of parameter space that can produce detectable bursts is small. The SNR shows a similar dependence upon periapsis as for the GC, and may be described by a power-law of
\begin{equation}
\log\left(\hat{\rho}\right) \simeq -2.7\log\left(\frac{r\sub{p}}{r\sub{g}}\right) + 3.1,
\end{equation}
for orbits with $r\sub{p} \gtrsim 10 r\sub{g}$. For a $1 M_\odot$ ($10 M_\odot$) object, bursts should be detectable for periapses $r\sub{p} \lesssim 7 r\sub{g}$ ($r\sub{p} \lesssim 14 r\sub{g}$), corresponding to $2.6 \times 10^{10}\units{m}$ ($4.9 \times 10^{10}\units{m}$) or $8.4 \times 10^{-7}\units{pc}$ ($1.6 \times 10^{-6}\units{pc}$). This leads us to conclude that extragalactic bursts are likely to be rare.


\chapter{The loss cone}

When considering the orbits of stars about a massive black hole (MBH), the loss cone describes a region of velocity space that is depopulated because of tidal disruption.

A main sequence star may be disrupted by tidal forces before it is swallowed by a MBH, we will define the tidal disruption radius as $r\sub{T}$. We would expect any orbit that passes inside $r\sub{T}$ to be depopulated unless stars may successfully escape to another orbit before being disrupted. Stars' velocities change because of gravitational interaction with other stars. Deflections can be modelled as a series of two-body encounters, the cumulative effect of which is a random walk in velocity space. Changes scale with the square-root of time, with the relaxation time-scale $\tau\sub{R}$ setting the scale.

Consider a typical star at a distance $r$ from the MBH. We can decompose its motion into radial and tangential components as
\begin{subequations}
\begin{align}
v\sub{r} = {} & v\cos \theta \\
v_\perp = {} & v\sin \theta.
\end{align}
\end{subequations}
Over a dynamical time-scale $t\sub{dyn}$, we expect that stars would change velocity by a typical amount
\begin{equation}
\theta\sub{D} \approx \left(\frac{t\sub{dyn}}{\tau\sub{R}}\right)^{1/2},
\end{equation}
assuming this change is small. We define the loss cone angle $\theta\sub{LC}$ to describe the range of trajectories that will proceed to pass within a distance $r\sub{T}$ of the MBH. By comparing the diffusion and loss cone angles we can deduce if we would expect orbits to be depleted: if $\theta\sub{D} > \theta\sub{LC}$ a star can safely diffuse out of the loss cone before it is destroyed, whereas if $\theta\sub{D} < \theta\sub{LC}$ a star will be disrupted before it can change its velocity sufficiently, leading to the depopulation of the orbit.

Frank and Rees first introduced the loss cone. They considered stars on nearly radial orbits. The orbital energy and angular momentum (per unit mass) of an object with eccentricity $e$ and periapse radius $r\sub{p}$ are
\begin{align}
\mathcal{E} = {} & -\frac{GM_\bullet(1 - e)}{2r\sub{p}} \\
\matcal{J}^2 = {} & GM_\bullet(1 + e)r\sub{p},
\end{align}
where $M_\bullet$ is the MBH's mass. The angular momentum can also be defined as
\begin{align}
\mathcal{J}^2 = {} & v_\perp^2r^2 \nonumber \\*
 \simeq {} & \theta^2v^2r^2,
\end{align}
using the small angle approximation. Frank and Rees took the limit $e \rightarrow 1$, then setting $r\sub{p} = r\sub{T}$ to demarcate the limit of the loss cone, we can rearrange to find
\begin{equation}
\theta\sub{LC} \simeq \frac{2GM_\bullet r\sub{T}}{v^2r^2}.
\end{equation}
We need to find the speed at $r$. Frank and Rees use a typical value
\begin{equation}
v^2 \simeq 3\sigma^2,
\end{equation}
where $\sigma$ is the 1D velocity dispersion. They assume that the velocity dispersion is Keplerian within the core region where dynamics are dominated by the MBH, and is a constant outside of this
\begin{equation}
\sigma^2 \simeq \begin{cases}
\frac{GM_\bullet}{r} & r < r\sub{c} \\
\frac{GM_\bullet}{r\sub{c}} & r < r\sub{c}
\end{cases}.
\end{equation}
Here the core radius $r\sub{c}$ is chosen such that
\begin{equation}
r\sub{c} = \frac{GM_\bullet}{\sigma_0^2},
\end{equation}
where $\sigma_0$ is the 1D velocity dispersion far from the MBH.

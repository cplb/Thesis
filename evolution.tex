\chapter{Evolution from gravitational wave emission}\label{ap:evolve}

Binary systems evolve under the influence of GW emission as energy and angular momentum are radiated away. They inspiral and eventually merge. For EMR systems, the evolution time-scale is typically long. We demonstrate this here by calculating the time taken to complete an inspiral in the case of a bound orbit and the change in orbital parameters over an orbit in the case it is unbound. A characteristic GW instantaneous evolution time-scale for bound orbits is also defined in \secref{GW-in}. These calculations assume Keplerian orbits, as found in weak-fields; corrections become more significant in stronger fields and eventually, at plunge, evolution proceeds rapidly. 

\section{Bound orbits}\label{ap:Bound}

For bound orbits, we can define a GW inspiral time from the orbit-averaged change in the orbital parameters. Using the analysis of \citet{Peters1964} for Keplerian binaries, the averaged rates of change of the periapsis and eccentricity are
\begin{align}
\left\langle\diff{r\sub{p}}{t}\right\rangle = {} & -\dfrac{64}{5}\dfrac{G^3M_\bullet M(M_\bullet + M)}{c^5r\sub{p}^3}\dfrac{(1 - e)^{3/2}}{(1 + e)^{7/2}}\left(1 - \dfrac{7}{12}e + \dfrac{7}{8}e^2 + \dfrac{47}{192}e^3\right) \\
\left\langle\diff{e}{t}\right\rangle = {} & -\dfrac{304}{15}\dfrac{G^3M_\bullet M(M_\bullet + M)}{c^5r\sub{p}^4}\dfrac{e(1 - e)^{3/2}}{(1 + e)^{5/2}}\left(1 + \dfrac{121}{304}e^2\right).
\end{align}
For a circular orbit, the inspiral time from initial periapsis $r\sub{0}$ is
\begin{equation}
\tau\sub{c}(r\sub{0}) = \dfrac{5}{256}\dfrac{c^5r\sub{0}^4}{G^3M_\bullet M(M_\bullet + M)}.
\end{equation}
For an orbit of non-zero eccentricity ($0 < e < 1$), we can solve for the periapsis as a function of eccentricity
\begin{equation}
r\sub{p}(e) = \mathcal{R}(1 + e)^{-1}\left(1 + \dfrac{121}{304}e^2\right)^{870/2299}e^{12/19},
\end{equation}
where $\mathcal{R}$ is fixed by the initial conditions: for an orbit with initial eccentricity $e_0$,
\begin{equation}
\mathcal{R}(e_0) = (1 + e_0)\left(1 + \dfrac{121}{304}e_0^2\right)^{-870/2299}e_0^{-12/19}r\sub{0}.
\end{equation}
The inspiral is complete when the eccentricity has decayed to zero; the inspiral time is \citep{Peters1964}
\begin{equation}
\tau\sub{insp}(r\sub{0},e_0) = \intd{0}{e_0}{\dfrac{15}{304}\dfrac{c^5\mathcal{R}^4}{G^3M_\bullet M(M_\bullet + M)}\dfrac{e^{29/19}}{(1-e^2)^{3/2}}\left(1 + \dfrac{121}{304}e^2\right)^{1181/2299}}{e}.
\end{equation}
This is best evaluated numerically, but it may be written in closed form as
\begin{align}
\tau\sub{insp}(r\sub{0},e_0) = {} & \tau\sub{c}(r\sub{0})(1 + e_0)^4\left(1 + \dfrac{121}{304}e_0^2\right)^{-3480/2299} \nonumber \\* 
 {} & \times \left. F_1\left(\dfrac{24}{19};\dfrac{3}{2},-\dfrac{1181}{2299};\dfrac{43}{19};e_0^2,-\dfrac{121}{304}e_0^2\right),\right.
\label{eq:Bound_inspiral}
\end{align}
using the Appell hypergeometric function of the first kind $F_1(\alpha;\beta,\beta';\gamma;x,y)$ \citep[16.15.1]{Olver2010}.\footnote{For small eccentricities, $\tau\sub{insp}(r\sub{0},e_0) \simeq \tau\sub{c}(r\sub{0})[1 + 4e_0 + (273/43)e_0^2 + \order{e_0^3}]$.}

\section{Unbound orbits}\label{ap:Unbound}

Unbound objects only pass through periapsis once. We therefore expect the orbital change from gravitational radiation to be small. Following the approach of \citet{Turner1977}, we can calculate the evolution in the eccentricity and periapse of an unbound Keplerian binary. The change in fractional eccentricity over an orbit, approximating the orbital parameters as constant, is
\begin{align}
\dfrac{\Delta e}{e} = {} & -\dfrac{608}{15}\Sigma\left[\recip{(1+e)^{5/2}}\left(1 + \dfrac{121}{304}e^2\right)\cos^{-1}\left(-\recip{e}\right) \right. \nonumber \\*  {} & + \left. \dfrac{(e - 1)^{1/2}}{e^2(1+e)^2}\left(\dfrac{67}{456} + \dfrac{1069}{912}e^2 + \dfrac{3}{38}e^4\right)\right],
\end{align}
introducing the dimensionless parameter
\begin{equation}
\Sigma = \dfrac{G^{5/2}M_\bullet M(M_\bullet+ M)}{c^5r\sub{p}^{5/2}}.
\end{equation}
Similarly, the fractional change in periapsis is
\begin{align}
\dfrac{\Delta r\sub{p}}{r\sub{p}} = {} & -\dfrac{128}{5}\Sigma\left[\recip{(1+e)^{7/2}}\left(1 - \dfrac{7}{12}e + \dfrac{7}{8}e^2 + \dfrac{47}{192}e^3\right)\cos^{-1}\left(-\recip{e}\right) \right. \nonumber \\*
 {} & - \left. \dfrac{(e - 1)^{1/2}}{e(1 + e)^3}\left(\dfrac{67}{288} - \dfrac{13}{8}e + \dfrac{133}{576}e^2 - \dfrac{1}{4}e^3 - \dfrac{1}{8}e^4\right)\right].
\end{align}
Both of these changes obtain their greatest magnitudes for large eccentricities, then
\begin{equation}
\dfrac{\Delta e}{e} \simeq \dfrac{\Delta r\sub{p}}{r\sub{p}} \simeq -\dfrac{16}{5}\Sigma e^{1/2}.
\end{equation}
For extreme mass-ratio binaries, as is the case here, the mass-ratio is a small quantity
\begin{equation}
\eta = \dfrac{M}{M_\bullet} \ll 1.
\end{equation}
The smallest possible periapsis is of the order of the Schwarzschild radius of the MBH, such that 
\begin{equation}
r\sub{p} = \alpha\dfrac{GM_\bullet}{c^2},
\end{equation}
where $\alpha > 1$. These give
\begin{equation}
\Sigma = \dfrac{\eta}{\alpha^{5/2}} < \eta \ll 1.
\end{equation}
Hence, the changes in the orbital parameters become significant ($\Delta e / e \simeq 1$) for
\begin{equation}
e \sim \dfrac{25}{256}\dfrac{\alpha^5}{\eta^2} > \dfrac{25}{256}\recip{\eta^2}.
\end{equation}
Such orbits should be exceedingly rare, and so it is safe to neglect inspiral for unbound orbits.

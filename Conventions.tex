\chapter{Conventions}
\markboth{Conventions}{}
\label{conventions}

Throughout this work we will use the time-like sign convention of \citet{Landau1975}:
\begin{enumerate}
\item The metric has signature $(+,-,-,-)$.
\item The Riemann tensor is defined as ${R^\mu}_{\nu\sigma\rho} = \partial_\sigma {\Gamma^\mu}_{\nu\rho} - \partial_\rho {\Gamma^\mu}_{\nu\sigma} + {\Gamma^\mu}_{\lambda\sigma}{\Gamma^\lambda}_{\rho\nu} - {\Gamma^\mu}_{\lambda\rho}{\Gamma^\lambda}_{\sigma\nu}$.
\item The Ricci tensor is defined as the contraction $R_{\mu\nu} = {R^\lambda}_{\mu\lambda\nu}$.
\end{enumerate}
Here ${\Gamma^\lambda}_{\mu\nu}$ are the Christoffel symbols. Greek indices are used to represent spacetime indices $\mu = \{0,1,2,3\}$ %(or $\mu = \{t,\widetilde{r},\theta,\phi\}$)
and lowercase Latin indices are used for spatial indices $i = \{1,2,3\}$. Uppercase Latin indices from the beginning of the alphabet are used to label detectors: $A = \{\mathrm{I}, \mathrm{II}\}$ for LISA, which has three arms and acts as two detectors, and $A = \{\mathrm{I}\}$ for eLISA, which has only two arms and so acts as a single detector. Lowercase Latin or Greek indices from the beginning of the alphabet are used for a more general parameter space. Summation over repeated indices is assumed unless explicitly noted otherwise.

We use $M_\odot$ and $R_\odot$ for the solar mass and radius, and $M_\oplus$ and $R_\oplus$ for Earth's mass and radius. The Minkowski metric is represented by $\eta_{\mu\nu}$. The natural logarithm is denoted by $\ln$ while the logarithm to base $10$ is $\log$.

In general, factors of the speed of light $c$ and gravitational constant $G$ are retained, except where explicitly noted. In chapters \ref{ch:f-R1} and \ref{ch:f-R2} we use natural units with $c = 1$. In sections \ref{sec:Geodesic} and \ref{sec:Energy-comp}, \chapref{resonances} and \apref{energy} we use geometric units with $G = c = 1$; a convenient conversion in this unit system is $10^6 M_\odot \simeq 4.92\units{s}$.

We use ``periapsis'' and ``apoapsis'' for the points of closest and furthest approach in an orbit about a black hole. \citet{Frank1976}, following the suggestion of Stoeger, introduced the terms ``-bothron'' as a substitute for the generic ``-apsis'' from the Greek {$\mathit{\beta\acute{o}\vartheta\rho o \varsigma}$}. In ancient Greek this refers to a sacrificial pit; however, in modern usage it is a cesspit.\footnote{John Eldridge has commented that either may be a suitable description of a Ph.D.} Therefore, we stick to the general case.

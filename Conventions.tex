\chapter{Conventions}
\markboth{Conventions}{}

Throughout this work we will use the time-like sign convention of \citet{Landau1975}:
\begin{enumerate}
\item The metric has signature $(+,-,-,-)$.
\item The Riemann tensor is defined as ${R^\mu}_{\nu\sigma\rho} = \partial_\sigma {\Gamma^\mu}_{\nu\rho} - \partial_\rho {\Gamma^\mu}_{\nu\sigma} + {\Gamma^\mu}_{\lambda\sigma}{\Gamma^\lambda}_{\rho\nu} - {\Gamma^\mu}_{\lambda\rho}{\Gamma^\lambda}_{\sigma\nu}$.
\item The Ricci tensor is defined as the contraction $R_{\mu\nu} = {R^\lambda}_{\mu\lambda\nu}$.
\end{enumerate}
Greek indices are used to represent spacetime indices $\mu = \{0,1,2,3\}$ %(or $\mu = \{t,\widetilde{r},\theta,\phi\}$)
and lowercase Latin indices are used for spatial indices $i = \{1,2,3\}$. Uppercase Latin indices from the beginning of the alphabet are used to label detectors: $A = \{\mathrm{I}, \mathrm{II}\}$ for \textit{LISA}, which has three arms and acts as two detectors, and $A = \{\mathrm{I}\}$ for \textit{eLISA}, which has only two arms and so acts as a single detector. Lowercase Latin indices from the beginning of the alphabet are used for parameter space. Summation over repeated indices is assumed unless explicitly noted otherwise. In general, factors of the speed of light $c$ and gravitational constant $G$ are retained, except where explicitly noted. 
